\documentclass[12pt,a4paper]{report}
\usepackage{geometry}
 \geometry{
 left=34mm,
 right=34mm,
 top=30mm,
 bottom=30mm,
 footskip=20mm
 }
\usepackage[english,russian]{babel}
\usepackage[T1]{fontenc}
\usepackage[utf8]{inputenc}
\usepackage{paratype}
\PassOptionsToPackage{hyphens}{url}
\usepackage{titlesec}
\titleformat*{\section}{\large}
\usepackage{paralist}
\usepackage{hyperref}
\usepackage{cleveref}[2012/02/15]% v0.18.4;
\setlength{\parskip}{1em}
\hypersetup{
    colorlinks=true,
    linkcolor=blue,
    urlcolor=blue,
}
\urlstyle{same}
\crefformat{footnote}{#2\footnotemark[#1]#3}

\title{%
	\textbf{Homo supersapiens}\\
	\large{Человек сверхразумный}}
\author{Александр Увижев\\
		uvizhe@yandex.ru\\
		supersapiens.org}

\date{2020}
\begin{document}

\maketitle

\begin{abstract}
Мы живем в эпоху, когда риск глобальной катастрофы растет с каждым годом, и мы знаем об этом. Но, несмотря на мрачные прогнозы ученых и философов, мы продолжаем эгоистично преследовать каждый свои собственные интересы. Человечество кажется не способным скоординировать усилия и контролировать экзистенциальные риски, и ситуация только усугубляется. Причиной такому положению вещей является сама природа жизни (а значит и наша человеческая природа), и мы не можем ее изменить. Тем не менее, существуют определенные знания, которые способны трансформировать личные убеждения человека и способствовать более просоциальному поведению и личному благополучию. Такие знания (и я утверждаю, что они есть ни что иное, как понимание природы жизни и собственной человеческой природы) являются основной целью в различных религиозных традициях, которые предлагают постигать их следуя сложным практикам. Если мы придумаем простой и быстрый способ приобретения таких трансформирующих знаний, то человечество станет альтруистичным (т.е., способным противостоять риску глобальной катастрофы) и более благополучным сообществом.
\end{abstract}

\chapter*{Человек разумный}

Всю культурную историю человечества нас сопровождают страшные сказки о конце света. Закат Пятого Солнца у Ацтеков, Рагнарёк в скандинавской мифологии, христианский Апокалипсис -- вот лишь несколько примеров. За свою жизнь почти каждый из нас успел пережить несколько таких предполагаемых судных дней. Последним из запомнившихся был конец света по календарю Майя в 2012 году, к которому многие всерьез готовились. Мы, похоже, настолько привыкли к подобному дискурсу, что не замечаем, как странные мистические и сектантские пророчества уступили место настойчивым предупреждениям светских мыслителей и ученых.

\noindent Одним из наиболее известных примеров таких тревожных сигналов является проект «Часы Судного Дня», символически показывающий близость возможной глобальной катастрофы. В 1945 году чикагские ученые, принимавшие участие в разработке атомной бомбы, основали журнал «Бюллетень ученых-атомщиков» и с тех пор каждый год, совместно с нобелевскими лауреатами, выставляют стрелки Часов. В 2018 году, впервые со времен холодной войны, Часы снова были выставлены на отметку 23:58, то есть в двух минутах от апокалисиса и переведены еще на 20 секунд ближе к «полуночи» в 2020 году.\footnote{\url{https://thebulletin.org/2020/01/press-release-it-is-now-100-seconds-to-midnight/}} Ученые отмечают огромные риски для нашей цивилизации, связанные с ядерным вооружением и глобальным потеплением, а также с ростом дезинформации в интернете.

\noindent Работы шведского философа Ника Бострома, посвященные экзистенциальным рискам человечества, также показывают довольно мрачную перспективу. В 2018 году вышла его очередная статья о высокой вероятности плохого конца технологической цивилизации.\footnote{\url{https://nickbostrom.com/papers/vulnerable.pdf}} Статья называется «Гипотеза уязвимого мира» и исследует возможность появления технологии, угрожающей миру катастрофой (напр. как ядерное оружие), невозможность избежать появления такого изобретения и вероятность самой катастрофы вследствие этого.

\noindent Будущее по-прежнему неизвестно, однако, в отличие от экстравагантных заявлений мистического толка, подобные аналитические оценки дают куда лучшее представление о вероятных сценариях развития событий, которых действительно стоит опасаться.\footnote{С более полным списком возможных угроз глобального масштаба можно ознакомиться тут: \url{https://en.wikipedia.org/wiki/Global_catastrophic_risk}} Похоже, нам нужно всерьез задуматься над нашим общим будущим и своим личным вкладом в него. Я же хочу поделиться своим видением того, как можно значительно снизить вероятность глобальной катастрофы и вместе с тем проложить путь к счастливому будущему человечества.

\noindent Поскольку наш сегодняшний мир, переживающий эпоху глобальной цивилизации -- достаточно сложная штука, я буду описывать его как систему, на каждом уровне детализации которой действуют свои законы. По мере раскрытия общей картины во всех необходимых подробностях, я покажу почему многие существующие идеи изменения мира являются нерабочими (глава \emph{«Человек разумный»}) и где нужно искать выход из ситуации (глава \emph{«Человек сверхразумный»}).

\section*{Мир в первом приближении:\\личные интересы против общих}

Может показаться странным, что вероятность катастрофы увеличивается вопреки нашему желанию. Однако, если посмотреть на мир людей с достаточного расстояния, становится понятно, почему так происходит.

\noindent Дело в том, что наш мир является продуктом деятельности множества независимых агентов. Каждое действие каждого человека на Земле формирует облик нашей цивилизации в той или иной степени. И нет никого, кто мог бы все контролировать. Каждый человек влияет на вероятность катастрофы.

\noindent Вместе с тем, каждый человек преследует свои интересы. Хотя предотвратить катастрофу -- в интересах каждого, но если выбор идет между умереть от голода завтра или погибнуть от возможной катастрофы в неопределенном будущем, любой человек, конечно, решит заняться добычей пищи. А уже потом, если будет возможность, подумать об общих и глобальных проблемах.

\noindent Из независимости действующих сил и приоритета личных интересов над общими возникает сложность координации. И вот мы неспособны сообща остановить глобальное потепление или гонку вооружений, потому что всегда заняты чем-то другим. Национальные политические лидеры, отвечающие за вопросы такого уровня, тоже преследуют свои интересы, важнейшими из которых могут быть удержание власти или улучшение жизни нации. Обе этих цели не связаны напрямую ни с климатом, ни с вооружением и часто даже идут вразрез с общемировыми потребностями. Например, добыча и сжигание ископаемого топлива, способствующие росту парниковых газов в атмосфере Земли, могут быть выгодны некоторым нациям, если это самый дешевый способ получать энергию на нужды страны.\footnote{Отчет о неиссякающей государственной поддержке наименее экологичной в производстве энергии угольной промышленности можно почитать тут: \url{http://priceofoil.org/2019/06/24/g20-coal-subsidies/}} Или некоторые страны могут разрабатывать новые виды вооружения из страха военного вторжения других государств и невозможности гарантировать, что кто-нибудь втайне не готовит такой план.\footnote{История ядерного оружия тому хороший пример: \href{https://ru.wikipedia.org/wiki/\%D0\%9C\%D0\%B0\%D0\%BD\%D1\%85\%D1\%8D\%D1\%82\%D1\%82\%D0\%B5\%D0\%BD\%D1\%81\%D0\%BA\%D0\%B8\%D0\%B9_\%D0\%BF\%D1\%80\%D0\%BE\%D0\%B5\%D0\%BA\%D1\%82}{https://ru.wikipedia.org/wiki/\\Манхэттенский\_проект}}

\noindent Кроме того в мире 600 миллионов людей все еще живут за чертой бедности.\footnote{\url{https://ourworldindata.org/extreme-poverty\#note-7}} Они почти все время заняты выживанием и у них просто нет времени подумать о последствиях своих действий и их влиянии на возможность катастрофы. Пластиковые стаканчики и соломки для питья являются роскошью в некоторых богатых странах, где за переработку мусора нужно платить. Но для огромного числа людей пластиковая тара -- это просто самый дешевый способ выживать -- одноразовая посуда гарантирует стерильность и при этом очень дешевая. И вот мы получаем тотальное загрязнение Земли пластиком и ничего не можем с этим сделать. 600 миллионов людей -- это огромная сила.

\noindent Может показаться, что остальные люди могли бы скоординировать усилия, сложить свои ресурсы и как-то решить эту проблему. Но, к сожалению, люди любого уровня достатка большую часть времени заняты тем же самым -- выживанием. Просто это уже не поиск пищи и денег, а усилия, направленные, на то, чтобы оставаться на своем уровне достатка -- не потерять работу, не потерять конкурентное преимущество, не потерять возможность ни в чем себе не отказывать. Население планеты стремительно растет, увеличивая конкурентное давление, и чтобы оставаться на месте нужно бежать со всех ног. Каждый сам за себя.

\noindent Также, огромный вклад в осложнение ситуации вносит прогресс. Прогресс невозможно остановить, ведь люди постоянно что-нибудь изобретают -- кто-то потому, что нравится, кто-то для того, чтобы не умереть от голода. Прогресс дарит нам все больше способов влиять на планету и делать это все эффективней. Разнообразные продукты химической промышленности от бытовой химии до гербицидов вроде Агента Оранж, мириады форм применения пластика, смартфоны и прочие гаджеты, которые мы пока не научились толком утилизировать. В итоге, с каждым годом мы производим все больше и быстрее, понимая масштаб последствий лишь постфактум. В этом разрушительном хаосе никто не имеет цели приблизить катастрофу -- просто каждый преследует свои интересы. Каждый хочет произвести и продать как можно больше конкурентноспособного товара, чтобы накормить себя и свою семью не хуже чем вчера.\footnote{Одна из иллюстраций к этому -- история про китайских производителей холодильников, использующих запрещенные хладагенты, ради сохранения бизнеса: \url{https://www.engadget.com/2018/06/27/investigators-china-illegal-cfc-emissions/}}

\section*{Привлекательная идея\\экономического равенства и достатка}

Кажется, что экономическое неравенство является серьезным фактором, множеством рук толкающим цивилизацию к пропасти. И оно, как будто, не только приближает катастрофу, но и порождает множество социальных проблем вроде нищеты, социальной напряженности и конфликтов из этого вытекающих. Естественно, на это обращали внимание многие и, конечно, идея всеобщего равенства и достатка неоднократно рождалась в обществе и выражалась то в философских трактатах, то бунтах и революциях. Однако, глядя на сегодняшний мир можно утверждать, что ни одна попытка реализовать эту идею не увенчалась окончательным успехом. Ведь равенство в игре, где каждый играет сам за себя едва ли возможно.

\noindent Обычно неравенство изображают в виде пирамиды. Конечно же, большинство тех, кто находится у ее основания стремятся подняться повыше, никто не хочет опускаться ниже, а те кто на самом верху используют все свое могущество, чтобы оставаться на своем месте. Когда происходит какая-нибудь революция, пирамида просто встряхивается и в разных ее местах оказываются новые люди -- пирамида не исчезает. Устранение неравенства относится к такому же классу задач, что и борьба с глобальным потеплением. Это не личная проблема, а общая и эффективно ее можно решать лишь сообща. Но если всегда есть люди, которым неравенство на руку, и они, обладая значительной властью, способны нивелировать многие достижения на этом фронте, то можно ли вообще мечтать об успехе?

\noindent Но, вероятно, равенство и не нужно. Может быть достаточно, чтобы все люди имели возможность вести достойную жизнь? Теоретически, человечество в будущем может дожить до эпохи всеобщего изобилия, когда всю необходимую работу будут выполнять машины, а люди будут лишь наслаждаться жизнью. Но даже если мы сможем этого когда-нибудь достичь, не уничтожив себя раньше, мы не сможем гарантировать равенство убеждений. Люди по-прежнему будут убивать из-за идей. А поскольку технологии становятся все доступнее и так разрушительны даже когда применяются без намерения навредить, то что уж говорить о том, что есть люди, которые хотят смерти множеству других людей. Катастрофа по-прежнему будет висеть над нами.

\noindent Почему удовлетворение базовых потребностей не избавляет людей от агрессии я покажу чуть позже, а пока краткий экскурс в устройство жизни и человека. Понимание этого нужно для дальнейшей детализации модели нашего мира.

\section*{Об устройстве жизни}

Жизнь существует благодаря способности организмов к самовоспроизводству. То есть, способности выживать и размножаться в данных условиях среды. Гены, которые передаются потомкам в процессе размножения, отвечают за все свойства организмов, а свойства обуславливают поведение. И это не только то, как живые организмы внешне реагируют на окружающий мир, но и то, как работают их клетки, органы, тело и мозг. Все свойства вместе с поведением определяют способность организмов к самовоспроизводству -- или можно сказать определяют вероятность выживания и размножения.

\noindent Из-за того, что ресурсы для самовоспроизводства ограничены, выживают и размножаются только самые приспособленные -- те, кто успешнее других получают необходимое в условиях конкуренции. Те, чье поведение и свойства, благодаря оптимальному набору генов, являются самыми эффективными для решения задач выживания и размножения. Наборы генов, которые решают какие-то еще задачи, просто не могут быть широко представлены в нашем мире и исчезают -- ведь они тратят часть энергии на какие-то еще вещи кроме самовоспроизводства, а значит уступают в нем тем, кто использует все доступные ресурсы только для этой задачи. Так работает естественный отбор.

\noindent Из-за того, что копирование и передача генов потомкам в процессе размножения работает неидеально, возникают мутации. Плохие мутации, из-за которых свойства и поведение организма приводят к уменьшению вероятности самовоспроизводства, быстро отсеиваются естественным отбором. Хорошие мутации, делающие набор генов более эффективным, наоборот -- из-за большей вероятности самовоспроизводства приводят к широкому распространению организмов с такой мутацией. Нетрудно понять, что при таком положении дел преобладающие наборы генов -- биологические виды -- с течением времени постоянно меняются, уступая место новым, более эффективным в текущих условиях среды. А среда меняется не только из-за климата, но и появления и распространения новых видов. И это суть эволюции.

\noindent Благодаря эволюции, и тому, что молекула ДНК достаточно маленькая, часть представленных в природе организмов обладает очень большим набором генов -- просто потому что это возможно. Эти большие наборы кодируют огромное множество свойств и очень сложное поведение. Если многоклеточные организмы не менее эффективны в самовоспроизводстве, чем одноклеточные, а животные с неокортексом не менее эффективны, чем без него -- значит они могут существовать в природе.

\noindent Тем не менее, все живые организмы, независимо от сложности, в общем виде -- это биологические машины, оптимизированные эволюцией и естественным отбором для задач выживания и размножения. И жизнь каждого существа можно представить в виде простой блок-схемы:

\begin{center}
Угроза жизни $\,\to\,$ Ответная реакция $\,\to\,$ Результат реакции
\end{center}

\noindent Угрозы жизни здесь -- это угрозы самовоспроизводству. Можно называть их стимулами -- они могут быть внешние или внутренние -- но в любом случае это те сигналы, которые имеют значение для выживания и размножения. Все другие сигналы не находят отклика в системе (организме) -- ведь это лишняя трата энергии и необходимых для реагирования свойств и поведения просто не заложено в генах. Поэтому можно называть все такие сигналы-стимулы, способные вызвать отклик, угрозами жизни. Будучи восприняты организмом, они вызывают ответную реакцию (отклик), приводящую к какому-то результату. Если после этого угроза жизни не устранена, все повторяется.

\noindent Поскольку не все угрозы жизни одинаково серьезны, то разные поведенческие реакции (или программы поведения) имеют разный приоритет. Например, нельзя одновременно убегать от тигра и кушать. Вероятность быть съеденным приближающимся хищником гораздо выше, чем вероятность умереть от голода. Поэтому животные обычно убегают. Так же поиск еды обычно важнее, чем поиск партнера для размножения (хоть и не всегда) или налаживание социальных связей.

\noindent Часто такое разбиение по приоритетам представляют в виде иерархии потребностей, и различные действия живых организмов подразумевают стремление к удовлетворению разных потребностей. Но, независимо от уровня абстракции и терминологии, важно не забывать, что все эти действия, какими бы сложными и отвлеченными они не казались, на самом деле являются реакциями на сигналы угрозы жизни. Потребности -- это просто набор целей, достижение которых обычно устраняет эти угрозы. Убегание от хищника спасает от смерти; еда спасает от голода; доминирование в стае спасает от необходимости искать партнера для размножения и от голода в случае нехватки еды. Удовлетворение разных потребностей, будь то собирание ресурсов или уклонение от прямой угрозы жизни\footnote{Под прямыми угрозами жизни я понимаю такие, которые не зависят от каких-либо ресурсов, например угроза здесь и сейчас быть съеденным хищником. Все остальные, косвенные угрозы жизни, зависят от доступности определенных ресурсов.} -- это ответные реакции на угрозы самовоспроизводства, которые в мире с ограниченными ресурсами возникают постоянно. У простых живых организмов это довольно прозрачно, но когда речь заходит о людях, часто кажется, что невозможно свести все наши странные поступки и желания к этой схеме.

\section*{Об устройстве человека}

Человек действительно устроен очень сложно. Именно эта сложность является причиной того, что часто мы не видим связи своего поведения с заложенной в генах функцией самовоспроизводства. Обычно мы замечаем лишь некоторую долю внешних раздражителей (стимулов) и внешнюю часть реакции. Все остальное остается вне поля нашего зрения, и человек, как бы, представляет собой невероятно загадочный черный ящик.

\noindent Это наш мозг, который даже для науки продолжает оставаться черным ящиком, всему виной. Он у нас настолько сложен, что мы гордо носим звание «человека разумного». Как я уже сказал, некоторые организмы довольно сложны и это оттого, что свойства ДНК позволяют наборам генов быть очень длинными и сложными. Увеличение числа свойств и усложнение поведения, во-первых, возможны, а во-вторых, иногда приводят к увеличению эффективности самовоспроизводства. И так, в процессе все большего накопления полезных мутаций, на свет появился распространенный набор генов Homo sapiens. Если бы мы имели самый чуткий нюх, идеальные слух и зрение, способности к камуфляжу, умели летать, быстрее всех бегать и производить по миллиону отпрысков -- мы бы назывались как-то иначе. Но полезные мутации были связаны в основном с устройством мозга и позволяли с помощью обретенного «разума» предвидеть больше гипотетических угроз жизни и так эффективнее им противостоять.

\noindent Благодаря нашему мозгу мы можем сохранять информацию о мире: иметь убеждения о нем и помнить собственный опыт -- в том числе прошлые угрозы жизни, наши реакции на них и результаты реакций. Благодаря мозгу и накопленной в нем информации мы можем моделировать в голове возможные варианты развития событий, просчитывать оптимальные реакции и так прогнозировать гипотетические угрозы жизни и быть готовыми к ним. Это дает нам прекрасное эволюционное преимущество -- способность менять свое поведение и адаптироваться к меняющейся среде не дожидаясь пока под действием естественного отбора изменятся наши гены. Подобные свойства мозга позволили нашему виду эффективно самовоспроизводиться, однако, на деле это означает, что процесс скрупулезного поиска гипотетических угроз и лучших реакций на них стал существенным элементом нашего поведения. Мы почти все время увлечены анализом прошлого опыта и моделированием будущего. Для мозга это задачи по умолчанию.\footnote{См. \href{https://ru.wikipedia.org/wiki/\%D0\%A1\%D0\%B5\%D1\%82\%D1\%8C_\%D0\%BF\%D0\%B0\%D1\%81\%D1\%81\%D0\%B8\%D0\%B2\%D0\%BD\%D0\%BE\%D0\%B3\%D0\%BE_\%D1\%80\%D0\%B5\%D0\%B6\%D0\%B8\%D0\%BC\%D0\%B0_\%D1\%80\%D0\%B0\%D0\%B1\%D0\%BE\%D1\%82\%D1\%8B_\%D0\%BC\%D0\%BE\%D0\%B7\%D0\%B3\%D0\%B0}{https://ru.wikipedia.org/wiki/Сеть\_пассивного\_режима\_работы\_мозга}}

\noindent Для того, чтобы это работало, нам нужно как-то ориентироваться в бессчетном множестве реальных и гипотетических угроз и всех возможных ответов на них. Поскольку в генах не прописана реакция на каждую возможную угрозу жизни, то нужно как-то понимать, какие сигналы важнее и какие реакции лучше. Поэтому кроме рефлексов и инстинктов у нас, как и у многих других животных, есть система вознаграждения/наказания, помогающая нам с этой задачей. Она воздействует нейромедиаторами на нашу нервную систему когда сигналы важны и когда наши реакции полезны или вредны для самовоспроизводства. И мы ощущаем это воздействие в виде чувств и эмоций, переживание которых, часто неосознаваемое, мотивирует нас действовать. Неприятные переживания -- пытаться их прекратить, приятные -- сделать так, чтобы они продолжались.

\noindent Это стремление к комфортным переживаниям определяет наше поведение везде, где в генах не прописано однозначной реакции. Сознательно или бессознательно мы оцениваем критичность стимула (не прямые угрозы жизни при этом представляются нам скорее как угрозы комфорту) и «вычисляем» наилучший вариант действий, отталкиваясь в решении этой задачи от качества переживаний, своих убеждений и опыта. И хотя такое поведение неплохо служит генам в задаче самовоспроизводства\footnote{Статистически это работает точно так же как любое распространенное поведение. Поведение возникает и распространяется, если в большинстве случаев помогает самовоспроизводству. Но подобно тому, как утоление голода может привести к отравлению, не всякое действие, вызванное желанием комфорта бывает полезно.}, результатом этого постоянного «переживания» реальности является то, что наши опыт и убеждения становятся как бы окрашены маркерами «плохо/хорошо» и затем это, в свою очередь, влияет на наше отношение к угрозам жизни и наши реакции на них.

\noindent Как видно, мы и правда устроены сложно, однако, несмотря на это, человек, похоже, является такой же биологической машиной, оптимизированной для самовоспроизводства и оперирует в рамках той же схемы, что и другие живые организмы. Вся наша жизнь -- это череда реакций на стимулы. Как минимум, можно смело утверждать, что большая часть поведения людей свойственна и многим другим видам и определяется генами, унаследованными от наших общих предков. Мы также проявляем агрессию, также как другие социальные виды стремимся к себе подобным, также стремимся доминировать среди них. Но, в отличие от многих других видов, мы реагируем не только на реальные угрозы жизни, но и на постоянно рождающиеся в нашем воображении, и наши реакции часто определяются не только нашими генами, но и оценочно-окрашенным отпечатком всего нашего опыта в нашем мозге.

\noindent Итак, кратко рассмотрев природу человека, можно перейти к дальнейшей детализации модели мира. Конечно, я коснулся не всего, что делает нас людьми, но для понимания последующего материала этого достаточно.

\section*{Мир во втором приближении:\\конкуренция за ресурсы}

Как любому живому организму, человеку нужны ресурсы для выживания и размножения. И в мире людей эти ресурсы так же ограничены, что приводит к конкуренции за них -- то есть к конфликту интересов. Желая разрешить этот конфликт мы пытаемся приблизить эпоху изобилия, чтобы базовые потребности каждого были удовлетворены. Но мы не понимаем, что ограниченность ресурсов -- это неотъемлемая часть феномена жизни. Когда вокруг изобилие, живым организмам ничто не мешает размножаться (в экспоненциальном порядке) пока ресурсы не станут ограничены снова. Именно поэтому вместе со стремительным прогрессом многократно увеличилось и население Земли. И хотя этот рост населения на единицу пространства вероятно когда-нибудь прекратится\footnote{Так нам предсказывает теория демографического перехода: \href{https://ru.wikipedia.org/wiki/\%D0\%94\%D0\%B5\%D0\%BC\%D0\%BE\%D0\%B3\%D1\%80\%D0\%B0\%D1\%84\%D0\%B8\%D1\%87\%D0\%B5\%D1\%81\%D0\%BA\%D0\%B8\%D0\%B9_\%D0\%BF\%D0\%B5\%D1\%80\%D0\%B5\%D1\%85\%D0\%BE\%D0\%B4}{https://ru.wikipedia.org/wiki/Демографический\_переход}}, проблема ресурсов здесь не заканчивается.

\noindent Мы обладаем большим числом различных стратегий поведения для добычи ресурсов, эволюционно унаследованных от предков: искать ресурсы самостоятельно; объединяться с другими особями, чтобы эффективнее добывать ресурсы; бороться за власть в сообществе, чтобы получать ресурсы с еще большей вероятностью и т.д. Но мы так изобретательны, что придумываем различные абстрактные сущности вроде денег или прав, на основе которых создаем совершенно новые стратегии. И тут уже можно говорить о культуре. Но какая бы она ни была, в результате мы просто получаем еще больше разнообразных желанных ресурсов, часть из которых довольно абстрактные, но могут быть обменены на нужные вещи. И вот вместе с материальными ресурсами мы ищем все эти абстрактные блага: деньги, лайки, статусы, позиции в рейтингах. Но и они в силу своей природы не могут существовать в достаточном для всех количестве.

\noindent Но и это еще не все. Даже если все потребности человека удовлетворены, наш мозг продолжает искать гипотетические угрозы жизни и находит их в непредсказуемом будущем. И мы начинаем запасаться ресурсами на «черный день». При такой размытой постановке задачи «необходимые ресурсы» просто не имеют предела.

\noindent Впрочем, объективно он, конечно, есть, ведь наша жизнь конечна. Наш уникальный мозг способен осознать этот факт\footnote{См. \url{https://en.wikipedia.org/wiki/Mortality_salience}} и тогда мы пытаемся придумать как решить эту проблему. Наверное, поэтому религии так популярны -- ведь они обещают вечную жизнь (а некоторые и максимальный комфорт после смерти), то есть предлагают простое решение проблемы. Однако, как бы ни были популярны религии, одной лишь верой устранить такую серьезную угрозу жизни как неизбежность смерти получается далеко не у всех. Более того, сами религии порождают множество конфликтов интересов, конкурируя с другими религиями и борясь с неверующими -- и поэтому не способны помочь ни с проблемой ресурсов ни с глобальной катастрофой.\footnote{Глобальную катастрофу многие религии вообще рассматривают как нечто неизбежное и даже благое.}

\noindent Рассматривая мир на уровне человечества и пытаясь понять, что способствует приближению катастрофы, мы обнаружили следующее. Люди стараются держаться на плаву, каждый на своем уровне достатка и желая большего. Эта личная проблема часто перевешивает в приоритете проблемы общие, вроде возможной катастрофы. Чем ниже уровень жизни, тем меньше возможности у человека переключиться с личных проблем на общие и снизить свой вклад в приближение катастрофы. Этот недостаток внимания и зачастую конфликтующие интересы не дают нам скоординировать усилия по предотвращению катастрофы, а технологический прогресс только усиливает негативные эффекты такого положения вещей.

\noindent Теперь мы понимаем, что стремление держаться на плаву и желать большего -- это неотъемлемая часть нашей природы, это борьба за ресурсы для удовлетворения потребностей, стремление к комфорту, которого всегда хочется только больше. Неравенство -- лишь следствие принципиальной ограниченности ресурсов, в условиях когда каждый играет за себя и не имеет планки «достаточно». Мы просто обречены на конфликт интересов, конкурируя за ресурсы в попытке удовлетворить свои безграничные потребности -- иначе говоря, постоянно реагируя на угрозы жизни, которые мы видим повсюду, но реже всего в конкуренции, этот конфликт усиливающей. Такова наша природа. Именно из нее возникают свойства мира, рассмотренные в первом приближении: независимость агентов, преследование каждым своих интересов и неминуемый прогресс. Мы независимы, потому что у каждого из нас свой набор генов, единственная задача которого -- самовоспроизводство. Мы преследуем собственные интересы по той же причине.\footnote{У каждого из нас собственный уникальный набор генов, ведь у всех нас разные родители. Каждый набор генов стремится воспроизводить себя, а значит конкурирует с другими наборами. Совсем иначе дело обстоит у общественных насекомых, где значительная часть колонии может происходить от одной пары, а значит иметь один набор генов. Поэтому, в отличие от них, в наших генах гораздо слабее выражены свойства и поведение, отвечающие за кооперацию. Кооперация не выгодна лучшим наборам генов так же как лидеру гонки не выгодно помогать отстающим.} Мы изобретаем, потому что это одна из возможных стратегий удовлетворения потребностей. Мы просто действуем в рамках нашего устройства и это порождает конфликты, агрессию\footnote{Агрессия -- это одна из стандартных реакций на угрозу жизни, позволяющая решать конфликты. Если конфликты неизбежны даже при удовлетворении базовых потребностей, то и агрессия неизбежна.} и конкуренцию, даже когда все базовые потребности удовлетворены.

\section*{Привлекательная идея хорошей культуры}

Я уже говорил, что, благодаря развитому мозгу, мы способны придумывать новые реакции на угрозы жизни -- иначе говоря, менять свое поведение. Конечно, не полностью, это не касается, например, безусловных рефлексов, но поведение на высоком уровне, вроде социального, доступно для коррекции. Это легко наблюдать, сравнивая людей, выросших в разных культурах или имеющих разные степени образования. И нам кажется, что культура и образование как ее важная часть являются ключом к решению многих проблем. Различные философии, идеологии и религии пытаются привить людям определенную культуру, которая, как предполагается, избавит общество от иерархий, соперничества, ненасытности, страха перед смертью и других пороков. Но удается ли им это, и может ли культура решить нашу задачу изменить мир так, чтобы риск катастрофы снизился?

\noindent Говоря о культуре на личном уровне, я подразумеваю совокупность убеждений отдельного человека и его шаблоны мышления и поведения, которые отчасти обусловлены убеждениями. Все это представляет собой созданные при жизни устойчивые нейронные связи в мозге и они устойчивы благодаря постоянному их использованию. Мы продолжаем мыслить и действовать шаблонно, опираясь на наши убеждения, если это решает определенные жизненные задачи. Люди живут в социуме, и условия в нем меняются порой гораздо быстрее, чем природа. Возможность создавать новые убеждения и шаблоны позволяет нам справляться с этой быстроменяющейся средой, не полагаясь на медленную эволюцию генов. Люди как бы подстраивают свой мозг под среду так, чтобы поведение не переставало быть эффективным. И вместе с тем, каждый со своим уникальным набором убеждений и шаблонов, сформированных уникальным личным опытом, мы влияем на нашу среду обитания и вместе формируем ее. Эти условия социальной среды, как результат совокупности культур каждого участника социума -- это культура в более общем смысле.

\noindent Однако, создавать устойчивые нейронные связи в мозгу довольно ресурсоемкая задача. Большая их часть создается в детстве, когда постигать и подстраиваться под правила мира -- наиболее важная задача. Большая часть культуры приходит к нам с воспитанием. Затем мы довольно неохотно меняем свои убеждения и шаблоны мышления\hspace{.05em}/\hspace{.05em}поведения, ведь мы оптимизированы эволюцией для максимально эффективной траты энергии. Проще действовать каждый день по известным сценариям и придерживаться своих убеждений. Мы не меняем свою жизнь ради новых привычек и не ломаем голову над новыми знаниями, пусть даже противоречат нашим убеждениям, если это кажется тратой энергии с непонятным результатом.\footnote{Изменение убеждений и шаблонов мышления\hspace{.05em}/\hspace{.05em}поведения является довольно существенным расходом энергии, и притом с плохо прогнозируемым результатом, ведь невозможно сказать, на что будет похожа наша жизнь с новыми нейронными связями и новыми реакциями на стимулы.} Только очень значимые стимулы имеют хорошие шансы нас изменить. Например, возникновение прямой угрозы жизни в случае, если мы не изменим своего поведения или мнения. Или когда мы видим серьезные плюсы в том, чтобы поменяться – например, когда носителями новой культуры становится довольно большая доля общества и начинает быть выгодно к ним присоединиться.\footnote{См. \href{https://ru.wikipedia.org/wiki/\%D0\%9A\%D0\%BE\%D0\%BD\%D1\%84\%D0\%BE\%D1\%80\%D0\%BC\%D0\%BD\%D0\%BE\%D1\%81\%D1\%82\%D1\%8C}{https://ru.wikipedia.org/wiki/Конформность}}

\noindent Именно поэтому образование, как инструмент изменения культуры, довольно неэффективно, ведь так велика разница между преподаваемыми знаниями и получаемыми. Образование подразумевает создание множества новых устойчивых нейронных связей. Мало кто готов на такое и очень редко условия среды этому способствуют. Даже доступное, бесплатное и обязательное образование не делает людей образованными автоматически. Так же и условия социальной среды, состоящей из множества элементов, не могут быстро прийти в соответствие с чьими-то идеями. Для кардинального изменения социальной среды, и культуры из нее рождающейся, требуется достаточно много времени, смена поколений, и лишь серьезное систематическое насилие, как нам не раз показывала история, может ускорить этот процесс.

\noindent Таким образом, любая попытка мирно направить цивилизацию в нужное русло через определенные культурные ориентиры, некие наборы убеждений и шаблонов мышления\hspace{.05em}/\hspace{.05em}поведения, сталкивается с инертностью человеческого существа и всего общества. Кроме того, в мире существует огромное множество представлений о том, как эти культурные ориентиры должны выглядеть. Сколько людей -- столько и мнений. Все эти убеждения о том, какие убеждения правильные, нередко противореча друг другу, конкурируют между собой, чем только усугубляют ситуацию.\footnote{Пожалуй, из-за подобного неравенства убеждений было пролито не меньше крови, чем из-за обычного желания добыть ресурсы.} Мы все отстаиваем свою культуру, охраняем и боремся за нее как за ресурс. Ведь она -- инструмент, помогающий жить и каждый день доказывающий нам свою эффективность.\footnote{Если я все еще жив -- значит инструмент эффективен.}

\noindent После всего сказанного, надеюсь, достаточно очевидно, что наша неспособность противостоять глобальным угрозам является не единственным нежелательным (хоть и самым серьезным) следствием человеческой природы. Множество проблем нашего мира вроде неравенства и нетерпимости, делающие жизнь большинства людей хуже, чем она может быть, также являются естественным результатом нашей природы.

\noindent В следующей главе я хочу показать, что, несмотря на невозможность изменить опасный курс цивилизации социально-экономическими методами, то есть через внешнее по отношению к человеку воздействие, все-таки у проблемы может быть решение. Такое решение основано на изменении социума и культуры человека не через внешние культурные ориентиры или экономические рамки, а через доступность определенных объективных знаний, рождающих общие для всех убеждения.\footnote{Подобно тому как знание о смертности человека делает убийство социально неприемлемым.}

\chapter*{Человек сверхразумный}

Рассматривая различные мистические и аскетические течения, появлявшиеся внутри религий и среди философских школ, можно заметить, что у них есть нечто общее. Хотя они полагаются на различающиеся внешне практики -- экстатические и опьяненные состояния, медитации, физические упражнения, аскетическое самоограничение и т.д. -- но все они предлагают человеку некоторый опыт, определенные трансформирующие переживания, недоступные в обыденной жизни.\footnote{Вообще, подобные практики в том или ином виде довольно давно и широко распространены: от ритуальных танцев в шаманизме, до современной телесно-ориентированной психотерапии. В основном, они, конечно, ассоциируются с религиозными традициями и в разной степени свойственны, наверное, всем религиям. Суфизм, дзэн, йога, христианский мистицизм опираются на индивидуальный опыт почти целиком.} Предполагается, что такой опыт позволяет адепту непосредственно понять определенные истины, избавляет от необходимости полагаться на слепую веру и направляет человека на верный путь.\footnote{Точно так же, мы все на своем опыте, а не из учебников, узнаем, что огонь обжигает.} Конечно, менять убеждения гораздо проще, когда видишь удивительные вещи своими глазами, чем с чужих слов.

\noindent Может показаться странным, что героям подобных культурных традиций, оформлявшим своей жизнью и опытом их очертания, часто приписываются похожие личные качества вроде скромности, смирения, милосердия, и одинаковые достижения, не характерные для большинства людей -- например, отказ от преследования материальных благ и трансцеденция эго. Но, видимо, дело в том, что, несмотря на разницу догм и практик, искомый трансформирующий опыт в первую очередь проистекает из общей для всех реальности -- и поэтому результат часто похож, хоть и может трактоваться по-разному в разных культурах. Конечно, нельзя отрицать, что может существовать такой опыт, который приводит к изменению человека в худшую сторону\footnote{Так, вероятно, человек, с пониженной способностью сопереживать, пройдя через опыт убийства и узнав как «легко» убивать людей, может стать хладнокровным убийцей.}, однако я хочу акцентировать внимание на той идее, что опыт познания реальности может сильно трансформировать человека.

\noindent Таким образом, можно сделать вывод, что определенный трансформирующий опыт познания реальности и вытекающий из него набор убеждений, то есть культура на личном уровне, могут быть общими для всех людей и доступны каждому. Они будут общими, потому что опираются на общую для всех реальность -- это и законы природы и тот факт, что мы принадлежим к одному биологическому виду и переживаем эту реальность почти одинаково. Они доступны каждому, потому что реальность доступна и потому, что такой трансформирующий опыт не опирается на какие-то предыдущие убеждения или условия социальной среды.

\noindent Ниже я хочу представить свою гипотезу о том, как опытное познание человеком собственной природы и природы жизни трансформирует убеждения и шаблоны мышления\hspace{.05em}/\hspace{.05em}поведения. И как это в свою очередь приводит к сокращению конфликтов интересов и увеличению вероятности кооперации, а значит и к снижению риска катастрофы. Человека, обладающего интуитивным пониманием\footnote{Тут я подразумеваю неявное знание (\href{https://ru.wikipedia.org/wiki/\%D0\%9D\%D0\%B5\%D1\%8F\%D0\%B2\%D0\%BD\%D0\%BE\%D0\%B5_\%D0\%B7\%D0\%BD\%D0\%B0\%D0\%BD\%D0\%B8\%D0\%B5}{https://ru.wikipedia.org/wiki/Неявное\_знание}), то есть основанное на опыте.} собственной природы и природы жизни и более просоциальным поведением вследствие этого я называю человеком сверхразумным.

\section*{Гипотеза о человеке сверхразумном}

Все люди хотят жить лучше. Это «лучше» формируется из наших концептуальных представлений о мире и себе, из наших убеждений. Еще часто используют термин «счастье» -- такой же неконкретный и субъективный. Эти «лучше» и «счастье» -- просто категории концептуального ума для оценки близости к точке максимального комфорта, помогающие нам принимать решения и действовать в условиях бесконечных комбинаций стимулов и возможных ответов на них.

\noindent В стремлении к счастью мы рисуем в своем воображении картину лучшей жизни, основанную на прошлом опыте и наших убеждениях о том, что «хорошо» и затем желаем ее достичь. Естественно, при этом нам хочется как-то оценивать эффективность наших действий, оценивать прогресс, то есть, текущую точку на шкале хуже-лучше. Особенно, когда нас постоянно спрашивают «как дела». И тут возникает сложность, ведь довольно непросто понять, стала ли жизнь комфортнее, чем месяц или год назад. Надо как-то сравнивать то ли переживания, то ли потребности, текущие и прошлые, но такое нашему уму не под силу. Нужен какой-то более простой критерий. Конечно, чем лучше он квантифицируем, тем проще нашему уму с ним работать -- надо просто складывать или сравнивать числа. Наверное, поэтому в нашей цивилизации так сильно развит феномен денег. Они символизируют возможность удовлетворения большинства потребностей и значит просто посчитав свой доход можно понять, правильно ли ты поступал и стал ли лучше жить.\footnote{Показательно, что мы до сих пор измеряем национальный и мировой успех в ВВП -- экономическом показателе, выражающемся в одной цифре. При этом с самого начала использования и по сей день многие указывают на то, что это плохой инструмент оценки социального прогресса.} И если да, то можно почувствовать себя счастливее. Сюда же относятся и прочие легко исчислимые и универсальные сущности, которые породила наша цивилизация -- этими ресурсами гораздо проще оперировать.\footnote{Например, довольно сложно вычислить степень уважения или власти -- гораздо проще посчитать и сравнить лайки или число подписчиков. Сложно оценить обеспеченность всей нужной едой или уровень комфорта, но довольно легко считать деньги, которые могут все это обеспечить. Автомобили, яхты, дома, позиции на карьерной лестнице так же легко пересчитываются в деньги или соотносятся друг с другом при сравнении своего и чужого.} Все эти символы благополучия в силу своей простоты служат нам удобным ориентиром в пути к лучшей жизни.

\noindent Теперь представим, что человек начинает осознавать бесконечность угроз жизни (т.е. потребностей) и видеть, что все его переживания вызваны ими. Это ведет к определенным выводам и соответствущему изменению поведения.\footnote{Тут речь скорее не про логические умозаключения, а про бессознательный выбор, подобно тому, как мы не суем руку в огонь. Хотя, человеку, обладающему знанием об обжигающем огне, кажется совершенно логичным остерегаться пламени, люди, не встречавшие ничего горячего в своей жизни, могут не увидеть в таком умозаключении смысла.} Так, становится очевидно, что путь к счастью -- это не конечный набор шагов, вроде «построить дом, посадить дерево, вырастить сына», а скорее бег в колесе. Потребности будут возникать всегда и их удовлетворение не ведет к их сокращению, а значит и путь к счастью не так прямолинеен. Конечно, можно пытаться нажить несметное богатство, чтобы удовлетворять вообще все потребности, но какой ценой? Ведь переживания, сопутствующие потребностям, качественно могут очень сильно различаться. Часто простые радости приносят больше положительных эмоций, чем умножение материальных благ. Часто действия, дающие конечную положительную сумму денег или удовлетворенных потребностей, оказываются довольно эмоционально негативными. И пусть в конце мы чувствуем определенную радость от формального успеха, но неприятный осадок сильно его нивелирует или даже перевешивает.

\noindent Получается, символам благополучия не всегда сопутствует комфорт. Даже формула «больше денег -- меньше проблем» работает не всегда, особенно после удовлетворения базовых потребностей. И когда оказывается, что общепринятые ценности не очень эффективны для достижения лучшей жизни, то в поисках новых, более подходящих, самому качеству переживаний начинает уделяться больше внимания. Тем более, что теперь у человека появляется больший контроль над ними.

\noindent Пока в удовлетворении потребностей мы невольно следуем каким-то выученным культурным шаблонам (вроде семья-карьера), попутно возникающие переживания могут восприниматься как внешние эффекты, на которые сложно повлиять. Но при смещении акцента на них становится очевидно, что не только другой приоритет удовлетворения потребностей может вести к большему комфорту, но даже другие реакции на угрозы жизни могут быть лучше. Ведь часто стандартная реакция (например, агрессия) является не самой эффективной и оборачивается плохими переживаниями.

\noindent В процессе поиска новых ориентиров, обращаясь к ощущениям, человек начинает замечать, как сильно на его переживания влияют другие люди. Например, то, как много эмоций порождает общение. Из этого приходит понимание того, как сильно люди влияют друг на друга. Чье-то плохое настроение может обернуться чем-то плохим для нас. А хорошее -- наоборот. Если при этом человек осознает собственную природу, то он перестает делить людей на плохих и хороших. Мы все лишь машины, реагирующие на стимулы, а значит действия всех людей обусловлены в основном обстоятельствами, текущими или прошлыми. Вследствие этих двух прозрений происходит переоценка значимости людей -- заботиться о них, то есть об их чувствах, становится гораздо важнее, чем раньше, и становятся важны все люди, независимо от того, как они далеки в любых культурных координатах. Ведь если каждый из нас так сильно обусловлен средой, в частности другими людьми, то любое позитивное взаимодействие улучшает эту среду и шансы получения приятных эмоций (и наоборот).

\noindent Вместе с этим человек осознает, что действовать сообща не только приятней, но и эффективней для каждого по отдельности. Создание общего блага -- лучшая стратегия достижения личного счастья, чем собирать ресурсы впрок или искать хорошей жизни только для себя. Во-первых, общее благо делает жизнь лучше здесь и сейчас, в то время как черный день может просто не настать. А даже если и настанет, то друзья, умножаемые в процессе кооперации -- самый универсальный ресурс. Во-вторых, стремление улучшать только свою жизнь, игнорируя окружающих, непременно приводит к контрасту между своим уютным мирком и становящейся все менее приятной всей остальной действительностью. Появляется необходимость в заборах, но они не уничтожают реальность, а лишь временно скрывают ее, и мир за забором становится все менее дружелюбным.

\noindent Получается, что с позиции эгоизма кооперация и альтруизм\footnote{Можно рассматривать альтруизм как стремление создать позитивную во всех смыслах обратную связь. Улучшение социальной среды ведет к большей вероятности переживания людьми положительных эмоций и дальнейшему улучшению социальной среды.} оказываются выгодны. Эволюционно механизмы кооперации в нас уже заложены -- но с тех пор как они появились, условия жизни сильно изменились. Мы сейчас живем не в маленьких изолированных группах, но в глобальном открытом пространстве, опирающемся на денежную экономику. Мы не доверяем незнакомым людям, которых теперь большинство вокруг нас и доверяем свое счастье деньгам -- из-за этого кооперативная стратегия часто уступает место конкуренции.\footnote{Мы боремся с незнакомыми и потому ничего не значащими для нас людьми за деньги и другие ресурсы точно так же как раньше конкурировали с другими видами и племенами.} Тем не менее, новые убеждения приводят к большей вероятности кооперативного поведения и сокращению числа конфликтов интересов, а значит и к снижению агрессии. Борьба за выживание начинает меньше походить на животную, когда потребности удовлетворяются почти рефлекторно, и становится более разумной, рациональной -- поскольку, видя общую картину, человек начинает принимать во внимание больше значимых переменных.\footnote{Разница между человеком разумным и человеком сверхразумным похожа на разницу между человеком, играющим в игру ради победы, и человеком, играющим ради удовольствия. Победу нельзя гарантировать (особенно, учитывая бренность жизни), но можно гарантировать удовольствие. Человек сверхразумный понимает, что конкурентная борьба за личное благо -- предприятие довольно сомнительное -- в то время как кооперативная игра ради общего блага практически гарантирует лучшую в плане переживаний жизнь.}

\noindent Общество людей сверхразумных будет гораздо менее агрессивным, чем наше нынешнее общество. Глобальные проблемы в нем будут решаться гораздо эффективнее. Ведь общие проблемы будут рассматриваться людьми сверхразумными как личные, а значит больше не будут игнорироваться и уступать им в приоритете, как это происходит сейчас.

\section*{Предпосылки гипотезы}

В начале этой главы я уже писал про традиции, опирающиеся на личный трансформирующий опыт. Одной из таких традиций является буддизм, который собрал, пожалуй, больше всего практической мудрости по этой теме.\footnote{Можно судить по времени существования традиции и числу адептов или по количеству созданных текстов и практик.} Изложенную выше гипотезу я строю отчасти на основе своих знаний о философии и практиках буддизма и своем представлении о том, как это все вместе работает.

\noindent Среди знаний, получаемых в этой традиции опытным путем и называемых «инсайтами», можно выделить три основных: «истину о непостоянстве», «истину о наличии и причинах страдания» и «истину о пустотности и отсутствии Я». Три эти «истины», будучи осознаны, дают достаточно хорошее понимание некоторых аспектов природы жизни и человека. Первая -- о нестатичности, постоянном изменении и эволюции любых феноменов в природе, что вполне укладывается в нашу модель мира, будь то звезды и атомы, биологические виды, свойства среды обитания или даже наши убеждения. То же можно сказать о потребностях и переживаниях, осознание непостоянства и неисчерпаемости которых вместе с их причинами и следствиями составляет второй «инсайт» о страданиях. Третья «истина о пустотности и отсутствии Я» является пониманием того, что вся природа и человек, как ее часть -- это механизм или система, работающая по определенным законам. В такой системе ничто не существует само по себе, все обусловлено механикой и динамикой системы. Буддизм заостряет внимание (и отсюда название истины) на том, что все субъективно наблюдаемые феномены -- лишь проявления состояния системы и не существуют самостоятельно, то есть пусты. В частности, не существует того, что мы называем личностью или «Я» (или душой и свободой воли), которое обладало бы какими-то независимыми хорошими или плохими качествами и действовало бы по своей воле -- то есть было бы независимо от системы. Это «Я» -- лишь следствие каких-то причин; все действия -- лишь реакции механизма.

\noindent В качестве основного инструмента постижения этих и других «истин» через собственный опыт буддизм использует практику медитации. С помощью медитации практикующий развивает в себе способность быть гораздо более внимательным к происходящему вокруг и внутри него -- в английских текстах обозначаемую как «mindfulness».\footnote{Я даю английский эквивалент термина, потому что большая часть современной литературы, посвященной этому вопросу, в том числе научных исследований, выходит на английском языке. На русский часто переводится как «осознанность», хотя мне кажется, что более подходящим переводом будет «внимательность».} Эта внимательность сама по себе тоже является инструментом для получения непосредственных трансформирующих знаний уже не только во время медитации, но и в обычной жизни.

\noindent Во время медитации человек получает возможность увидеть нескончаемый поток постоянно меняющихся угроз жизни и вызываемых ими переживаний. С развитием внимательности взору открываются постоянные переживания и в обычной жизни. Вместе с тем, внимательность позволяет увидеть собственную реактивность и импульсивность, ведь очень часто мы реагируем автоматически даже не замечая этого. Становится очевидно, что такие автоматические реакции, это наше выученное или врожденное поведение, часто являются довольно неуместными и приводят к новым переживаниям. Так становится понятна важность переживаний -- ведь они и причина и результат наших действий. И в то же время, видя собственные переживания и их причины, человек начинает понимать, что все его действия, мысли, чувства и эмоции, и в общем-то все черты его личности, берут начало в текущем или прошлом опыте. Приходит осознание собственной обусловленности внешними по отношению к «Я» обстоятельствами. А если так, то это «Я», или то, что называется «личностью», со всеми приписываемыми хорошими или плохими качествами оказывается лишней сущностью в картине мира.\footnote{В буддизме для обозначения всей совокупности прошлого опыта, обуславливающего человека, используется понятие «кармы». Чтобы увидеть существенную разницу, проистекающую из такого мировоззрения, сравните два утверждения и их возможные следствия: 1) он поступил так, потому что он плохой человек; 2) он поступил так, потому что у него плохая карма.} Так, начиная лучше понимать собственную природу, к человеку приходит лучшее понимание других людей -- таких же реактивных, обусловленных обстоятельствами, не хороших и не плохих. И вместе с этим пониманием приходят терпимость и сочувствие так свойственные буддизму.

\noindent Выше я коротко описал свою идею относительно того, каким образом работает буддийская традиция, благодаря многолетней практике и получаемому через него трансформирующему опыту, превращающая обычного человека в человека сверхразумного. Вместе с тем, хочется подкрепить свои размышления научными свидетельствами, и ниже я приведу ряд аргументов, основанных на экспериментальных данных.

\section*{Научные свидетельства}

Основания для гипотезы о том, что познание человеком собственной природы приводит к просоциальному поведению и является залогом позитивных изменений в социуме можно найти и в научных исследованиях. Главным аргументом здесь являются исследования, посвященные тесно связанным концепциям эмоционального интеллекта, эмпатии и perspective-taking.\footnote{Эмоциональный интеллект -- способность распознавать свои и чужие эмоции и использовать эту информацию; эмпатия -- это умение понимать чужие эмоции и их возможные причины; perspective-taking (\url{https://en.wikipedia.org/wiki/Perspective-taking}) означает способность взглянуть на мир глазами другого человека.} Несмотря на отсутствие строгих и общепринятых определений и довольно большую размытость этих понятий (особенно, когда дело доходит до измерений и лабораторных исследований), считается, что все эти способности завязаны на понимание человеком самого себя и применение этого знания по отношению к другим.\footnote{В основе такого мнения лежат изыскания, посвященные «модели психики человека» (\href{https://ru.wikipedia.org/wiki/\%D0\%9C\%D0\%BE\%D0\%B4\%D0\%B5\%D0\%BB\%D1\%8C_\%D0\%BF\%D1\%81\%D0\%B8\%D1\%85\%D0\%B8\%D1\%87\%D0\%B5\%D1\%81\%D0\%BA\%D0\%BE\%D0\%B3\%D0\%BE_\%D1\%81\%D0\%BE\%D1\%81\%D1\%82\%D0\%BE\%D1\%8F\%D0\%BD\%D0\%B8\%D1\%8F_\%D1\%87\%D0\%B5\%D0\%BB\%D0\%BE\%D0\%B2\%D0\%B5\%D0\%BA\%D0\%B0}{https://ru.wikipedia.org/wiki/Модель\_психического\_состояния\_человека}) (англ. Theory of Mind).} То есть, похоже, люди с лучшим пониманием собственной природы обладают бóльшим эмоциональным интеллектом, более сильными эмпатией и способностью понять чужую перспективу -- иначе говоря, лучше предсказывают других людей.

\noindent Исследования показывают, что эти связанные навыки коррелируют с просоциальным поведением\footnote{\label{one}Chopik, W. J., O’Brien, E., \& Konrath, S. H. (2017). Differences in Empathic Concern and Perspective Taking Across 63 Countries. \emph{Journal of Cross-Cultural Psychology, 48}(1), 23-38. \url{https://doi.org/10.1177/0022022116673910}} и другими положительными вещами. Так, например, они ассоциируются с лучшими социальными связями\cref{one}\textsuperscript{,}\footnote{\label{two}Mayer, J. D., Roberts, R. D., \& Barsade, S. G. (2008). Human Abilities: Emotional Intelligence. \emph{Annual Review of Psychology, 59}, 507–536. \url{https://doi.org/10.1146/annurev.psych.59.103006.093646}}, удовлетворенностью жизнью (т.е. счастьем)\cref{one}\textsuperscript{,}\cref{two} и даже как будто помогают снизить агрессию.\footnote{Day, A., Mohr, P., Howells, K., Gerace, A., \& Lim, L. (2011). The Role of Empathy in Anger Arousal in Violent Offenders and University Students. \emph{International Journal of Offender Therapy and Comparative Criminology, 56}(4), 599–613. \url{https://doi.org/10.1177/0306624x11431061}}\textsuperscript{,}\footnote{Vachon, D. D., \& Lynam, D. R. (2015). Fixing the Problem With Empathy. \emph{Assessment, 23}(2), 135–149. \url{https://doi.org/10.1177/1073191114567941}} Из этого можно заключить, что люди, которых я обозначил сверхразумными, каким-то образом более счастливы, интегрированы в общество и оказывают на него положительное воздействие.

\noindent Недавняя научная статья\footnote{Radzvilavicius, A. L., Stewart, A. J., Plotkin J. B. (2019). Evolution of empathetic moral evaluation. \emph{eLife, 8}, e44269. \url{https://doi.org/10.7554/eLife.44269}} утверждает, что за лучшей социальной интеграцией стоит все та же эмпатия. Люди, которые достаточно объективно воспринимают других меньше опираются на общественное мнение с распространенными в нем предубеждениями и охотнее идут на контакт с кем бы то ни было. И этот контакт очень важен, ведь мы социальные животные. Мы давно понимаем, что общение делает людей счастливее\footnote{На эту тему есть куча литературы, вот некоторые значимые исследования:
\begin{compactitem}
  \item Kahneman, D., Krueger, A. B., Schkade, D. A., Schwarz, N., \& Stone, A. A. (2004). A Survey Method for Characterizing Daily Life Experience: The Day Reconstruction Method. \emph{Science, 306}(5702), 1776–1780. \url{https://doi.org/10.1126/science.1103572}
  \item Diener, E., \& Seligman, M. E. (2002). Very Happy People. \emph{Psychological Science, 13}(1), 81–84. \url{https://doi.org/10.1111/1467-9280.00415}
  \item Ryan, R. M., \& Deci, E. L. (2001). On Happiness and Human Potentials: A Review of Research on Hedonic and Eudaimonic Well-Being. \emph{Annual Review of Psychology, 52}(1), 141–166. \url{https://doi.org/10.1146/annurev.psych.52.1.141}
\end{compactitem}\vspace{-1.3em}}; мы обнаружили, что счастье распространяется через социальные связи словно вирус.\footnote{Fowler, J. H., \& Christakis, N. A. (2008). Dynamic spread of happiness in a large social network: longitudinal analysis over 20 years in the Framingham Heart Study. \emph{BMJ, 337}, a2338. \url{https://doi.org/10.1136/bmj.a2338}} Мы видим, что социальная интеграция снижает агрессию в обществе и способствует просоциальному поведению.\footnote{Abrams, D., Hogg, M. A., \& Marques, J. M. (Eds.). (2005). The social psychology of inclusion and exclusion.}\textsuperscript{,}\footnote{Hamid, N., \& Pretus, C. (2019, June 12). The neuroscience of terrorism: how we convinced a group of radicals to let us scan their brains. \emph{The Conversation}. \url{https://theconversation.com/the-neuroscience-of-terrorism-how-we-convinced-a-group-of-radicals-to-let-us-scan-their-brains-114855}}

\noindent Выходит, лучшее понимание себя и, как следствие, других людей непосредственным образом благотворно сказывается на социальной среде. Сверхразумные люди не только сами более счастливы, но и способствуют счастью других людей -- просто повышая средний уровень счастья и социальных связей в обществе. А чем более мы счастливы -- то есть удовлетворены жизнью здесь и сейчас -- и чем больше у нас друзей, тем меньше необходимости в конкуренции и конфликтах. Ведь удовлетворенность -- есть отсутствие потребностей и необходимости бороться за ресурсы. А друзья -- это те, с кем можно добывать ресурсы вместе и почувствовать себя частью чего-то большего: коллектива, сообщества, планеты, о которой мы можем вместе заботиться.\footnote{Еще одно из исследований намекает, что то ли ощущение себя частью чего-то большего, то ли понимание, что все в мире взаимосвязано (обусловлено и едино), а может и то и другое, очень сильно коррелирует со счастьем:
\begin{compactitem}
  \item Edinger-Schons, L. M. (2019). Oneness beliefs and their effect on life satisfaction. \emph{Psychology of Religion and Spirituality.} \url{https://doi.org/10.1037/rel0000259}
\end{compactitem}\vspace{-1.3em}}

\section*{Недостающая деталь}

Итак, похоже, существуют некоторые универсальные и доступные каждому через личный опыт знания\footnote{Которые, в отличие от знания о горячем огне получают далеко не все люди.}, меняющие культуру человека в лучшую сторону. Такие трансформирующие знания могут сделать людей менее агрессивными, более просоциальными и счастливыми. А с ростом числа таких людей будет снижаться вероятность глобальной катастрофы, ведь будет меньше конфликтов интересов и больше кооперации ради общего блага. Однако, эта привлекательная перспектива сама по себе не способна стать реальностью и этим не сильно отличается от любой другой утопии. Поэтому, чтобы считать данный текст завершенным с моей точки зрения, следует показать возможность достижения описанной картины и необходимые шаги по ее реализации.

\noindent Прежде всего, следует заметить, что в феномене человека сверхразумного есть все необходимое для эффекта снежного кома. Во-первых, вероятность трансформации человека из обычного состояния в сверхразумное больше, чем в обратную сторону. Это можно сравнить с тем, как происходит переход к повсеместному использованию смартфонов -- по-новому просто субъективно лучше жить.\footnote{В наши дни лучше иметь смартфон, чем не иметь, ведь он позволяет пользоваться многими плодами современной цивилизации – иными словами, дает преимущество. Со сверхразумностью дело обстоит примерно так же, ведь новые убеждения и поведение с большей вероятностью делают человека счастливей.} Во-вторых, с ростом доли сверхразумных людей увеличиваются и степень кооперации, и прирост общего блага, и усилия, направленные на трансформацию всего общества. При достижении какой-то критической отметки все это приведет к положительной обратной связи, снежный ком покатится сам и со временем все люди станут сверхразумными.

\noindent Однако, чтобы запустить этот процесс преобразования не хватает одной вещи -- возможности быстро получать трансформирующий опыт. Если бы для приобретения смартфона человеку нужно было в течении многих лет тратить часть своего времени на странные ритуалы, вряд ли мы бы сейчас жили в эпоху смартфонов. Но с приобретением трансформирующих знаний дела обстоят именно таким образом.\footnote{Возвращаясь к примеру с огнем -- это знание получить легко и никому не приходит в голову спорить об огне. А, например, знание о том, что Земля круглая получить на личном опыте технически очень сложно. Мы все в основном верим в это, опираясь на чужие свидетельства, а некоторые даже и не верят.} Совсем немногие достаточно мотивированы, чтобы серьезно заниматься, к примеру, медитацией -- это сложно, непонятно и отнимает так много ценного в наш век стремительной жизни ресурса времени.

\noindent Если мы придумаем простой и быстрый способ получения необходимых для трансформации знаний, сравнимую по легкости с покупкой смартфона или поднесением руки к огню, то процесс изменения общества будет запущен и описанная мной перспектива вероятнее всего станет реальной достаточно быстро. Но для этого нам нужно достаточно хорошо разобраться в механике такой трансформации. Понять какие именно нужны знания и опыт\footnote{Выше я писал о неких опытных знаниях, буддийских инсайтах и эмпатии, однако мы не знаем как много общего между ними. Существующих сейчас исследований недостаточно, чтобы уверенно говорить о том, как медитация (или, скажем, йога) связана с эмпатией и насколько похожи возникающие эффекты. Тем не менее, само наличие такой связи почти несомненно и нам нужно картографировать эту неизведанную территорию.}, как определенные практики приводят к подобному опыту, как он влияет на наш мозг и как вообще работает наш мозг.

\noindent Это не простая область исследований и именно поэтому мы до сих пор так мало знаем о предмете. Но благодаря техническому прогрессу у нас появляется все больше возможностей заглядывать внутрь мозга. И хотя не все трудности (в том числе и методологические) еще устранены, но исследования в этой области постепенно набирают обороты. Вероятно, до «волшебной пилюли» нам еще очень далеко, но не стоит думать, что наша задача -- мгновенно делать всех сверхэмпатичными или просветленными супергероями, подобно Будде. Даже повышение среднего уровня эмпатии или внимательности в обществе на каких-нибудь 10\% уже может привести к значительным изменениям и запустить эффект снежного кома. И это не кажется какой-то заоблачной целью -- уже есть достаточно свидетельств перспективности некоторых подходов и технологий, и наверное, в скором времени мы можем начать собирать низко висящие плоды, если приложим достаточно усилий.\footnote{Хочу озвучить несколько идей, лежащих на поверхности, чтоб все это не казалось пустым звуком. Сейчас есть интересные исследования, показывающие, что можно довольно легко улучшать внимание: например, с помощью биологической обратной связи(i), прямой стимуляции ритмов работы мозга(ii) или даже просто играя на компьютере(iii). Мы можем пробовать провоцировать необходимый опыт, усиливая внимание каким-либо образом. Вероятно, даже с помощью носимых устройств в повседневной жизни. Тут может пригодиться и технология дополненной реальности (AR), а в будущем нейрокомпьютерные интерфейсы, вроде Neuralink. Виртуальная реальность (VR) также выглядит довольно перспективной технологией, обладая потенциалом для повышения эмпатии(iv)(v). Как мы видим, благодаря современным технологиям можно делать много интересного. Так, если можно улучшить эффективность медитации с помощью простого интерактивного приложения(vi), то чего можно добиться если использовать искусственный интеллект, большие данные и биологическую обратную связь?\\
Отдельно стоит упомянуть о психоделиках. Эти вещества давно известны человечеству и используются для получения недоступного в обычном состоянии сознания опыта уже не одно тысячелетие. Оказывая долговременный психологический эффект даже при однократном применении(vii), они все больше привлекают внимание ученых. Вместе с тем, известные своей способностью временно стирать личные границы, усиливая ощущение связи с миром и другими людьми, они могут, вероятно, служить хорошим инструментом стимуляции эмпатии и альтруизма. Интересно, что в недавней статье(viii) на \emph{The Conversation} излагается похожая идея «трансформирующего опыта» единения с природой, и автор задается вопросом, могут ли психоделики, усиливая этот опыт, помочь в решении экологических проблем нашего мира.
\begin{compactenum}[i)]
  \item Bagherzadeh, Y., Baldauf, D., Pantazis, D., \& Desimone, R. (2020). Alpha Synchrony and the Neurofeedback Control of Spatial Attention. \emph{Neuron, 105}(3), 577-587. \url{https://doi.org/10.1016/j.neuron.2019.11.001}
  \item Siegle, J. H., Pritchett, D. L., \& Moore, C. I. (2014). Gamma-range synchronization of fast-spiking interneurons can enhance detection of tactile stimuli. \emph{Nature Neuroscience, 17}(10), 1371–1379. \url{https://doi.org/10.1038/nn.3797}
  \item Patsenko, E. G., Adluru, N., Birn, R. M., Stodola, D. E., Kral, T. R., Farajian, R., ... \& Davidson, R. J. (2019) Mindfulness video game improves connectivity of the fronto-parietal attentional network in adolescents: A multi-modal imaging study. \emph{Scientific Reports, 9}(1), 1-8. \url{https://doi.org/10.1038/s41598-019-53393-x}
  \item Slater, M., Neyret, S., Johnston, T., Iruretagoyena, G., de la Campa Crespo, M. Á., Alabèrnia-Segura, M., ... \& Feixas, G. (2019). An experimental study of a virtual reality counselling paradigm using embodied self-dialogue. \emph{Scientific Reports, 9}(1), 1-13. \url{https://doi.org/10.1038/s41598-019-46877-3}
  \item Herrera, F., Bailenson, J., Weisz, E., Ogle, E., \& Zaki, J. (2018). Building long-term empathy: A large-scale comparison of traditional and virtual reality perspective-taking. \emph{PloS one, 13}(10), e0204494. \url{https://doi.org/10.1371/journal.pone.0204494}
  \item Ziegler, D. A., Simon, A. J., Gallen, C. L., Skinner, S., Janowich, J. R., Volponi, J. J., ... \& Gazzaley, A. (2019). Closed-loop digital meditation improves sustained attention in young adults. \emph{Nature Human Behaviour, 3}, 746–757. \url{https://doi.org/10.1038/s41562-019-0611-9}
  \item Nichols, D. E. (2016). Psychedelics. \emph{Pharmacological Reviews, 68}(2), 264–355. \url{https://doi.org/10.1124/pr.115.011478}
  \item Adams, M. (2020, January 28). Could psychedelics help us resolve the climate crisis? \emph{The Conversation}. \url{https://theconversation.com/could-psychedelics-help-us-resolve-the-climate-crisis-129639}
\end{compactenum}\vspace{-1.3em}}

\noindent Конечно, можно просто откинуться на спинку кресла и ждать, но если мы хотим быстрее снизить вероятность катастрофы и повысить шансы собственного выживания (а также поскорее достичь лучшей жизни), то в наших интересах приблизить тот день, когда наука сможет дать интересующие нас ответы. А значит, нам нужно больше нейроученых, больше профильных исследований, больше мест и программ, которые таких специалистов готовят и делают подобные исследования возможными. И для этого, как бы банально это ни звучало, нужно больше денег. Но это хорошая новость, ведь это значит, что любой человек, даже никак не связанный с наукой, легко может содействовать этому прогрессу. Так, частные пожертвования, обретая форму грантов, могут обеспечить дополнительные исследования и подготовку новых специалистов даже если основные патроны науки -- государство и крупный бизнес -- недостаточно заинтересованы в этом.\footnote{Чтобы сбор и инвестирование денег шли наиболее эффективно, кажется разумным создать отдельный фонд или институт, занимающийся задачей развития целевых исследований. Такая организация может не только работать с финансами, но также разрабатывать исследовательские гайдлайны и учебные курсы, вести собственную научную работу, взаимодействовать с государством и бизнесом для привлечения крупных инвестиций и лоббирования необходимых политических решений.} Подобная помощь в развитии нейронауки и целевых исследований должна стать приоритетной благотворительностью для каждого, кто обеспокоен возможностью катастрофы или более очевидными проблемами этого мира. Множество таких проблем имеют один корень и общее решение и попытка устранить их по отдельности подобна борьбе с симптомами, но не с самой болезнью. А чтобы победить «болезнь», надо стать разумнее -- стать Homo supersapiens.

\end{document}