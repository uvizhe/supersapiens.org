\documentclass[12pt]{report}
\usepackage{geometry}
 \geometry{
 left=36.5mm,
 right=36.5mm,
 top=36mm,
 bottom=32mm,
 footskip=20mm
 }
% \geometry{
% left=36mm,
% right=36mm,
% top=36mm,
% bottom=32mm,
% footskip=20mm
% }
% \geometry{
% left=35mm,
% right=35mm,
% top=30mm,
% bottom=30mm,
% footskip=20mm
% }
%\usepackage{newcent}
\usepackage{bera}
%\usepackage{newpxtext}
%\usepackage{stix}
%\usepackage{txfonts}
\usepackage[utf8]{inputenc}
\usepackage[english]{babel}
\usepackage[T1]{fontenc}
\PassOptionsToPackage{hyphens}{url}
\usepackage{titlesec}
\titleformat*{\section}{\large}
\usepackage{paralist}
\usepackage{hyperref}
\usepackage{cleveref}[2012/02/15]% v0.18.4;
\setlength{\parskip}{1em}
\hypersetup{
    colorlinks=true,
    linkcolor=blue,
    urlcolor=blue,
}
\urlstyle{same}
\crefformat{footnote}{#2\footnotemark[#1]#3}

\title{\textbf{Homo Supersapiens}}
\author{Alexander Uvizhev\\
		uvizhe@yandex.ru\\
		supersapiens.org}

\date{2020}
\begin{document}

\maketitle

\begin{abstract}
There is a growing chance of global catastrophe, as modern scientists and philosophers warn us. It seems that we are unable to coordinate our efforts and manage existential risks and make things even worse. The cause of this is the very nature of life (and so our human nature as well), and we can't change it. However, there is a kind of knowledge that can transform personal beliefs and lead to more pro-social behavior and subjective well-being. This knowledge (and I argue that it is a deep understanding of life and our human nature) is the holy grail in various religious traditions, and they offer to acquire it with hard to follow practices. If we create an approach for quick and easy acquisition of this transformative knowledge, then we transition to a cooperative society (which is capable of coordinating to cope with global catastrophic risk) and make everyone happier.
\end{abstract}

\chapter*{Sapiens}

Throughout human history we have been haunted by horrific tales of the end of the world. The Fifth Sunset in Aztec culture, Ragnarök in Norse mythology, and the Christian apocalypse\thinspace---\thinspace these are just a few examples. In today’s day and age, many of us have lived through alleged doomsdays, including the Mayan calendar’s prediction of the end of time (2012), which many people took seriously. Looks like we’ve become so much accustomed to this paranoid discourse that we don’t realize how these odd mystical and sectarian prophecies give way to insistent warnings from secular thinkers and scientists.

\noindent One of the most well-known examples of such warnings is the Doomsday Clock project, a symbol metaphorically representing the likelihood of a man-made global catastrophe as proximity to midnight. In 1945, Chicago scientists, who had participated in the development of a nuclear weapon, founded the magazine \textit{Bulletin of the Atomic Scientists}, and since then its members along with Nobel Prize Laureates set the clock every year. In 2018, for the first time since the Cold War, the clock was again set to two minutes to midnight (the largest value in its history at that time), and then twenty seconds closer to midnight in 2020.\footnote{\url{https://thebulletin.org/2020/01/press-release-it-is-now-100-seconds-to-midnight/}} Scientists point to huge threats of nuclear weapons, global warming, and Internet disinformation for our civilization.

\noindent The work of Swedish philosopher Nick Bostrom on existential risk also shows grim prospects. In 2018, he published a paper about the high probability of the collapse of any technological civilization.\footnote{\url{https://nickbostrom.com/papers/vulnerable.pdf}} Titled ``The Vulnerable World Hypothesis'', the paper explores the potential emergence of devastating technology (e.g., nuclear weapons), the inability to avoid such inventions, and the likelihood of a catastrophe as a result.

\noindent The future is still unknown, yet in contrast to extravagant mystical statements, such analytical assessments give us a much better idea of what really to beware.\footnote{A more comprehensive list of possible global-scale threats can be found here: \url{https://en.wikipedia.org/wiki/Global_catastrophic_risk}} It seems that we should start thinking more seriously about our shared future and our personal contribution to it. Here I want to share my view on how we can drastically minimize the possibility of a global catastrophe, and pave the way toward a better future at the same time.

\noindent The world is quite complex thing, so I will outline it in stages, presenting the internal driving forces on each level of detail. As the big picture unfolds I’ll show why many existing ``world-changing'' ideas don’t work (this chapter), and where to look for a possible solution (chapter ``Supersapiens'').

\section*{The World at First Glance: Personal vs Common Interests}

It may seem strange that the chance of catastrophe rises despite our will to prevent it. However, looking at the human world from some distance gives us a way to understand why this happens.

\noindent Our world is the product of a multitude of independent agents. Every action of every human being on the Earth shapes civilization in one way or another. There’s nobody to control it all. Every person influences the likelihood of a cataclysm, whether intentionally or not.

\noindent At the same time, every person pursues self-interests. One would think that preventing a catastrophe is in everyone’s best interest, yet when there’s a choice between starving tomorrow and perishing in a possible cataclysm some time in the indefinite future, we usually decide to look for food first. And once our more immediate needs are met, there is a chance we might think about common global problems.

\noindent Independence of actors and the priority of personal interests over the common good give rise to a coordination problem. And so we are unable to stop global warming or the arms race collectively\thinspace---\thinspace we’re always busy with other things. National leaders, who are responsible for resolving this kind of issues, also serve their own interests, primarily either holding onto power or improving the life of a nation’s people. Both of these goals are not directly connected to the climate or to weapons, and sometimes they even go against global needs. For example, the extraction and burning of fossil fuels, which contributes to greenhouse gas emissions causing global warming, can be advantageous for some nations if it’s the cheapest way to get energy for the needs of a country.\footnote{The report about inexhaustible state subsidies of the least eco-friendly energy-producing coal industry can be read here: \url{http://priceofoil.org/2019/06/24/g20-coal-subsidies/}} Or states may develop new weapons because they want to mitigate the risk of a military invasion and can’t warrant it any other way.\footnote{The history of a nuclear weapon is a good example: \url{https://en.wikipedia.org/wiki/Manhattan_Project}}

\noindent On the other hand, there are 600 million people in the world living below the poverty line.\footnote{\url{https://ourworldindata.org/extreme-poverty\#note-7}} Each of these people likely prioritizes survival most of the time, so they don’t have the luxury to think about how their actions may contribute to a global catastrophe. Plastic cups and straws might be an expensive choice in first-world countries where citizens have to pay for recycling. But for many people, plastic dishware is simply the cheapest way to survive: single use dishes and packaging ensure sterility and are very inexpensive. This unfortunately contributes to plastic pollution, which has now become an unmanageable problem. Collectively, six hundred million people produce a great impact on the planet.

\noindent One may think the rest of humanity could coordinate their efforts, pool resources, and solve the world’s problems somehow. Unfortunately, people at any level of wealth are preoccupied with the same thing\thinspace---\thinspace survival. It means not only searching for food and money, but also an effort to stay wealthy\thinspace---\thinspace making sure to not lose a job, not lose competitive advantage, not lose the capability to not deny yourself anything. The world population is growing, which increases competitive pressure, so we must run as fast as we can just to stay in place. Each is on their own.

\noindent Another force in the complex dynamics of the world is technological progress. It cannot be stopped; people keep inventing either because they like to or simply to get paid. Technological progress gives us endless new ways to affect the planet more extensively: a multitude of industrial chemical products from household cleaners to Agent Orange, myriad forms of plastic usage, and smartphones and other gadgets we are still unable to recycle properly. As a result, we produce more stuff, and faster every year, and we may only understand the scale of consequences for our planet post facto. In this devastating chaos most people do not want to intentionally cause catastrophe, but everybody pursues their own interests. Everyone wants to produce and sell as much competitive product as possible to be able to provide for themselves and their family no worse than yesterday.\footnote{The story of Chinese refrigerator manufacturers is a good illustration of this. They still use banned refrigerants to save their businesses: \url{https://www.engadget.com/2018/06/27/investigators-china-illegal-cfc-emissions/}}

\section*{The Attractive Idea of Economic Equality and Prosperity}

Economic inequality is really a big problem: it creates numerous social problems such as poverty, social tension, and conflicts and pushes our civilization into an abyss. Sure, this notion has bothered many minds. The idea of universal equality and prosperity has repeatedly emerged in society, giving birth to philosophical treatises, riots and revolutions. However, looking at today’s world, it’s clear that no attempt to bring the idea of economic equality to life has been truly successful. That’s because it’s hardly feasible to reach stable parity in a game where everyone plays for one’s self.

\noindent Inequality is usually represented as a pyramid. Of course, most people in this hierarchy try to lift themselves: nobody wants to go lower, and those on top use their power to stay in place. When a revolt happens the pyramid gets shaken up, and while people may find themselves in new positions the hierarchy does not disappear. Fighting inequality is a problem of the same class as fighting climate change: it affects everyone, and solving it requires a coordinated effort from all parties. And yet, if there are always people taking advantage of inequality, people who possess power so great that many achievements for the greater good can be negated, how can we dream of success at all?

\noindent Well, maybe equality isn’t necessary? Maybe, it would be enough if everyone could live a decent life? In the future we theoretically could live in the world of universal abundance, with machines doing all the necessary work, and people doing nothing but enjoying life. Nonetheless, even if we approach this future before we destroy ourselves, we cannot ensure the parity of beliefs. People will still kill because of different viewpoints. And because technologies are becoming more and more accessible and may wreak havoc even unintentionally, is it worth mentioning that ``some men just want to watch the world burn''? The danger of catastrophe will still loom over us.

\noindent I will discuss later why fulfilling basic human needs won’t cure aggression, and meanwhile a brief excursion into the machinery of life and human beings. This is to help us build a more detailed model of the world.

\section*{How Life Works}

Life exists because organisms can self-reproduce, which means they have the ability to survive and reproduce in a given environment. Genes that are passed on to offspring in the process of reproduction define all properties of an organism, and these properties determine behavior. All properties along with behavior determine an organism’s ability to self-reproduce, in other words, their likelihood of surviving and reproducing.

\noindent Because resources in a given environment are limited, only the most adapted organisms (and their genes) survive and reproduce. Those organisms that have greater chances to acquire vital resources under competitive conditions; those whose behavior and properties, thanks to genes, are the most efficient to meet the challenge of survival and reproduction. Genes that solve other tasks simply can’t be widespread and disappear, because they expend energy on tasks other than self-reproduction, and lose to those allocating all available resources to this very task. This is how natural selection works.

\noindent The process of replication and transmission of genes to offspring doesn’t work perfectly, and this causes mutations. Bad mutations, resulting in properties and behavior that decrease the likelihood of self-reproduction are eliminated naturally. In contrast, good mutations that make a set of genes more efficient result in a prevalence of organisms with those genes. It’s easy to see how this leads to a world where dominant sets of genes, species in other words, constantly change over time, giving way to new species that are more efficient in a current environment. Because the environment changes all the time as well, not just due to climate but also because of the emergence and proliferation of new species. This is the essence of evolution.

\noindent Due to evolution and DNA small dimensions some organisms possess very large sets of genes, just because it’s possible. The large sets can define a huge variety of properties and very complex behavior. If multicellular organisms can be as efficient in self-reproduction as unicellular organisms, and animals with neocortex can be as efficient as animals without it, then they can exist in the nature.

\noindent Nevertheless, all living organisms in general, and regardless of their level of complexity, are biological machines optimized by evolution and natural selection to survive and reproduce. As such, the life of any living being can be represented as a simple flow diagram:

\begin{center}
Life threat $\,\to\,$ Response reaction $\,\to\,$ Result of reaction
\end{center}

\noindent Life threats here mean threats to self-reproduction. We may call it stimuli, external or internal, no matter, it’s the signals meaningful for survival and reproduction. Any other signal cannot elicit a response because it’s a waste of energy, and genes simply don’t define properties and behavior to act upon it. This is why we may call any signal-stimulus which elicits a response a life threat. Once perceived by an organism it causes a response reaction and if this doesn’t eliminate the threat as a result then the process repeats.

\noindent Not all threats are equally dangerous, and so different behavioral responses have different priorities. For example, it’s hardly possible to flee from a tiger and eat at the same time. The chances of being killed by the approaching predator are much higher than the chances of dying of hunger. That’s why organisms usually flee in such situations. Similarly, foraging for food is usually more important than looking for a mate or establishing social bonds.

\noindent This differentiation of priorities is sometimes represented as hierarchy of needs, and different actions imply striving to meet different needs. But regardless of abstraction level and terminology it’s important to keep in mind that all actions, no matter how sophisticated or irrelevant they look, are actually response reactions to various threat signals. Needs are just a set of goals, that are likely to eliminate threats when approached. Fleeing from a predator saves an organism from immediate death, food saves it from starvation, being a dominant animal in a social group gives more opportunities for mating and getting a food. Satisfying different needs, whether gathering of resources or evading a direct threat\footnote{By direct threats I mean those which don’t involve any resources, like the threat to be eaten by a predator here and now. All other, indirect live threats, are tied to availability of certain resources.}, is just a response reaction to life threats that constantly appear in the world with limited resources. It’s pretty clear in the case of simple living beings, but when it comes to humans it’s often hard to boil down our weird acts and desires to this scheme.

\section*{How a Human Being Works}

Human beings indeed are rather complex organisms. That’s why we don’t usually attribute our behaviors to the genetic function of self-reproduction\thinspace---\thinspace we miss the link. We mostly see just a small portion of external causes (stimuli) and the observable reaction. Anything else is out of sight, and so humans appear mysterious black boxes.

\noindent The cause of this is the human brain that stays mysterious even for science. Our brains are so sophisticated that we proudly call ourselves ``Homo sapiens''. As previously mentioned, some organisms are rather complex, and this is due to DNA properties allowing for long genomes. An increased number of properties and behavioral complexity are possible, and sometimes they lead to more efficient reproduction. And thus, as a result of accumulation of good mutations, our Homo sapiens set of genes has developed. If we had the most sensitive abilities for scent detection, hearing, sight; or if we could fly, run like a cheetah and produce a million offspring, we’d have a different name than Homo sapiens. Rather, good mutations benefited the constitution of our brains and gave us sapience, allowing us to anticipate more hypothetical threats and counter them more effectively.

\noindent Human brains can retain various information from observations: hold beliefs and remember past experiences\thinspace---\thinspace including past threats, responses, and their results. With the help of a brain and accumulated knowledge we can simulate in our minds different courses of events, plan optimal reactions, and so foresee hypothetical threats and be prepared to face them. This gives us a great evolutionary advantage\thinspace---\thinspace the ability to change our behaviors and adapt to evolving environment before our genes change from natural selection. Though these advanced brain properties helped our species to proliferate, this also means that the meticulous search for hypothetical threats became an essential element of our behavior\thinspace---\thinspace we analyze past experiences and simulate the future almost constantly.\footnote{See \url{https://en.wikipedia.org/wiki/Default_mode_network}}

\noindent For mental analysis to happen we need to differentiate between countless threats\thinspace---\thinspace real and hypothetical\thinspace---\thinspace and between any possible reactions. Because every possible reaction isn’t set in the genes, we have to recognize which stimulus is the most important and which reaction is the most beneficial. That’s why as many other animals, along with reflexes and instincts, we have reward and punishment systems in the brain that help us with this task. These influence our nervous system with neurotransmitters when stimuli signals are important, and when any particular reaction would be useful for survival and reproduction. We perceive this influence as feelings and emotions, which motivate us to act, and often unconsciously. Unpleasant feelings make us try to cease them; pleasant ones motivate us to keep those feelings going.

\noindent This seeking of comfortable experiences always determines our behavior unless there’s another reaction defined in our genes, and we don’t perceive indirect life threats as life threats at all\thinspace---\thinspace instead, they look to us like threats to comfort. Consciously or not, we estimate the seriousness of stimuli and ``compute'' the best reaction based on our feelings’ subjective quality, our beliefs, and past experience. Though this way of functioning serves well for genes’ self-reproduction\footnote{Statistically this works like any other widespread behavior. A behavior emerges and spreads if it favors self-reproduction for a majority of the time. However, like an impulse to satisfy a hunger may lead to food poisoning, not every action evoked by comfort-seeking is beneficial.}, the presence of pleasant and unpleasant feelings, that accompanies many experiences leads to the condition where part of our knowledge (i.e., past experience and beliefs) is biased with labels of ``bad/good'', and this in turn influences our attitudes towards threats and our response reactions.

\noindent As we have seen, human beings are indeed very complex. Nevertheless, we’re ultimately biological machines like other species, optimized for self-reproduction and operating according to the same stimulus-response scheme. Our behavior is shared with many other species and was mostly inherited from our ancestors; we similarly show aggression, we strive to build kinship like other social animals, and we likewise try to attain a higher social rank. But unlike many other species, we respond not only to real threats but also to those that our imagination constantly generates, and our reactions are mostly determined by the bias of our cumulative past experiences rather than by genes.

\noindent So, after this brief excursion to life and human nature, we can now proceed to further refine our world vision. Sure, I didn’t touch all aspects of our nature, but that’s enough to understand further content.

\section*{The World at Second Glance: Competition for Resources}

Like any living organism, a human needs resources to survive and reproduce. Since they are limited, this leads to competition for resources\thinspace---\thinspace in other words conflict of interests. We seek to resolve this conflict, so we try to bring closer the abundance era in which the basic needs of every human being are met. However, we don’t understand that resource constraint is an integral part of the life phenomenon. If there’s abundance around, living organisms proliferate exponentially until resources become limited again. This is why the world’s population multiplied drastically along with the swift advance of technology. Although this growth of population density will probably end one day\footnote{As the theory of demographic transition predicts: \url{https://en.wikipedia.org/wiki/Demographic_transition}}, the problem of resource limit is broader in fact.

\noindent Humans have a variety of behavioral strategies for acquiring resources, which were inherited via evolution from our ancestors: seek resources on our own, cooperate with others for greater efficiency, fight for power in a social group in order to have more opportunities to meet our needs, and so on. Unlike other animals, we’ve invented various abstractions like money and rights, laying the basis for new behavioral strategies. We can start speaking of culture here, but regardless of its properties, the result is simply a greater number of distinct, desirable resources, some of which are quite abstract, yet can be exchanged for necessary things. So, along with material resources like food and shelter we seek all these abstract goods: money, likes, social rank, etc. But still, due to their nature of being resources they are subject to shortages, there isn’t enough for everyone.

\noindent On the other hand, even if all human needs are met, the brain continues its search for hypothetical threats and finds them in the unpredictable future. And we start to stockpile resources for a rainy day. Given this, it’s clear that the ``sufficient resources'' are actually boundless.

\noindent However, fairly speaking, there is a natural limit because our lives are finite. Our unique brain can conceive this fact\footnote{See \url{https://en.wikipedia.org/wiki/Mortality_salience}}, and then we try to solve this problem. This is probably why religions are so popular: they promise eternal life or even maximum comfort after death, thus giving a simple solution. Nevertheless, despite their popularity, religious faith alone isn’t enough for many of us to reconcile with such a big threat as inevitable death. Moreover, religions themselves cause many conflicts of interest\thinspace---\thinspace competing with each other and fighting heresy\thinspace---\thinspace and hence are unable to help either the resource problem or the chance of global catastrophe.\footnote{Many religions see a global catastrophe as something imminent and even good.}

\noindent When we initially examined why the possibility of a catastrophe arises, we saw the following: people try to maintain their wealth and acquire more. This individual problem often outweighs the priority of common problems like a potential catastrophe. The lower the quality of life, the fewer opportunities that a person has to focus on global issues or to reduce their own contribution to a forthcoming disaster because they are prioritizing their survival and individual needs. This lack of attention and often conflicting interests continue to prevent coordinated efforts to combat a global cataclysm. Technological progress at the same time amplifies the negative effects of the present condition.

\noindent Now we see that this effort to stay afloat and the desire for more are both integral parts of human nature. It’s the fight for resources to satisfy needs. It’s the comfort-seeking\thinspace---\thinspace and the more comfort, the better. Inequality is just a consequence of fundamental resource constraints, where every person has no threshold to know they have enough. We are doomed for conflicts of interest, competing for resources in an effort to meet our endless needs. Speaking generally, we constantly react to threats we see everywhere, but least of all in the very competition that amplifies the conflicts. It is our nature and the cause of the world’s properties that are visible at first glance: agents’ independence, the serving of personal interests, and unavoidable technological progress. We’re independent because everyone possesses their own set of genes with the only function of self-reproduction. We serve our own interests for the same reason.\footnote{Everyone has a unique set of genes because we all have different parents. Each set of genes self-replicates and hence competes with others. A very different case are species whose brood share the same set of genes, like most ants. Sure, human genes have less cooperative properties than ant genes, because cooperation isn’t beneficial for competing sets of genes, just as it isn’t beneficial for an individual runner to help outsiders.} We invent because it’s just one more strategy to satisfy our needs. We simply act according to the laws of our nature, and this creates conflicts, aggression\footnote{Aggression is the standard response to threats, helping to resolve conflicts. But if conflicts are unavoidable even after satisfying basic needs, then aggression is inevitable as well.}, and competition even if all of our basic needs are met.

\section*{The Attractive Idea of a Good Culture}

I mentioned previously that due to humans’ advanced brain we can invent new reactions to threats and thus change our behavior. Not completely of course, we cannot change our reflexes, for instance, but learned behavior like social reactions can be altered. This can be seen when comparing people from different cultures or different educational backgrounds. So we think that both culture and education are the keys to the solutions to many problems. Various philosophies, ideologies and religions try to instill specific cultures in us, which are supposed to free society from hierarchies, competitiveness, greed, fear of death, and other human vices. But do such efforts succeed? And is it possible that by fostering a culture meant to positively change the world we can lower our chance of a global catastrophe?

\noindent When referring to ``culture'' I mean a set of beliefs of any person, their thoughts and behavior patterns that are partly conditioned by beliefs. Thoughts and behavior correspond to stable neural connections in the human brain created throughout our lives, and they are stable because we use them routinely. We keep using particular patterns of thought and behavior and rely on our beliefs if they solve particular life challenges. However, humans live in a society that usually changes much faster than a natural environment. Being able to form new beliefs and thought/behavior patterns allows our species to cope with fast-changing environments and do not depend on the slow evolutionary process. We simply adjust our brains to changing conditions in order to keep our behavior efficient. And at the same time, each of us with our unique beliefs and thought/behavior patterns shaped by individual experience has an effect on our environment. This social environment is a collective culture in a more general sense, though it is the result of a combination of individual cultures.

\noindent However, the process of creating stable neural connections requires a lot of resources. Most of the stable connections are created in childhood when the human brain is developing and begins to understand and adapt to the rules of life. Most of a culture comes to us with nurture. Then we change our beliefs and patterns of thought/behavior quite reluctantly, because we are evolutionary optimized for the most efficient expense of energy. It’s much easier to stick to patterns and beliefs if changing would be a waste of energy for an unpredictable result\footnote{Changing beliefs and patterns of thought/behavior would require a rather substantial expense of energy with a hardly predictable result, because we cannot tell what our life will look like with new neural connections and new responses to stimuli.}, so we tend to not change our routines for the sake of new habits, not puzzle ourselves to acquire new knowledge, even if it contradicts our beliefs. Only the most significant stimuli provide a good motivation to change: a risk to our lives if we don’t change or if we see a great advantage in changing (e.g., a new culture becomes prevalent and joining this culture is beneficial\footnote{See \url{https://en.wikipedia.org/wiki/Conformity}}).

\noindent That’s why education as a tool to change culture is quite inefficient because there is an important difference between knowledge taught and knowledge learned. Education implies generating a lot of new stable neural connections, but few people willingly do this, and the environment rarely encourages it. Even education that’s accessible, free, and compulsory doesn’t make people educated automatically. The same with a social environment\thinspace---\thinspace composed of many elements it cannot rapidly match someones’ views. A radical change of social conditions and a resulting culture take a long time to build, and only severe systemic violence, as history has shown many times, may speed this process.

\noindent Thus, any attempt to peacefully guide a civilization toward a better direction with the help of specific cultural references (i.e., some sets of beliefs and patterns of thought/behavior) encounters the inertia of human nature and the whole society and halts. Besides that there are many different views on what the cultural references should be. ``Many men, many minds.'' All these views about the ``right'' views, often contrary to each other, compete and make situations even worse.\footnote{Perhaps a disagreement of viewpoints has caused more blood spilled than the thirst for resources.} Each of us advocates for own culture, defends it and fights for it like it’s a precious resource, because it’s a tool that helps us to live and proves each day its utility.\footnote{If I’m still alive then the tool is efficient.}

\noindent After all that’s been said here I hope it’s pretty obvious that our inability to cope with possible global threats is not the only undesirable (albeit the worst) consequence of human nature. Many problems in the world, like inequality and intolerance, which make people’s lives worse than they should be, are also the natural outcome of our human nature.

\noindent In the next chapter I will show that despite the inability to change the dangerous course of our civilization with socioeconomic methods, that is, with influences external to human beings, the problem nevertheless may have a solution. This solution is based on changing the society and an individual culture not through the external cultural references or economic constraints but rather with the accessibility of specific objective knowledge that leads to shared beliefs.\footnote{Like how the knowledge of human mortality makes murder socially unacceptable.}

\chapter*{Supersapiens}

Looking at various mystic and ascetic movements that emerged from religious traditions and philosophical schools, one may notice that these movements all have something in common. Although they rely on different practices\thinspace---\thinspace ecstatic and intoxicated states, meditation, physical exercise, ascetic self-restriction, etc.\thinspace---\thinspace they all offer certain transformative experiences usually inaccessible in our daily life.\footnote{Generally speaking, such practices in one form or another became widespread long ago, from shamanistic ritual dances to modern-day body-oriented psychotherapy. However, they are mostly associated with religious traditions and are probably inherent in all religions. Sufism, Zen, Yoga and Christian mysticism rely on individual experience almost entirely.} It’s supposed that such experiences allow an adept to comprehend particular truths, stop depending on blind faith, and see the right path to live.\footnote{Similarly, we can only know the heat of fire from our own experience, not from textbooks.} Of course, it’s much easier to change your beliefs if you see incredible things yourself, rather than relying on hearsay.

\noindent It may seem strange that the legendary people who shaped these different cultural traditions are usually ascribed with similar personal qualities like humility or benevolence as well as achievements uncharacteristic of most people, such as renouncing the pursuit of material goods or transcending the ego. However, despite difference in dogma and practice, this is probably because the sought transformative experience comes from the same reality, and so the result (i.e., achieved personal characteristics) is often similar no matter how it’s interpreted in different doctrines. Of course it’s undeniable that there’s possibly an experience that changes a person for the worse\footnote{It can likely be said that a person with little compassion can become a stone-cold killer after having gone through an experience involving murder and learning how it is ``easy'' to kill.}, but I want to focus on the idea that understanding reality (e.g., the nature of life and humans) from experience can significantly transform a person.

\noindent So we may conclude that certain transformative experiences that may come from a contact with reality, and the resulting set of beliefs (i.e., individual culture), can be the same for all people and are accessible to everyone. These are the same because experiences originate from the same reality like laws of nature or the fact that we belong to the same species and perceive the world in a similar way. These are accessible because reality is accessible, and because the transforming experiences don’t necessarily rely on prior beliefs or social conditions.

\noindent Below I want to present my hypothesis on how an experiential understanding of human nature and the nature of life transforms personal beliefs and patterns of thought/behavior, and how this in turn leads to a decrease of conflicts of interest and an increased likelihood of cooperation, which means a lower risk of a global catastrophe. Human beings that possess the intuitive knowledge\footnote{Here I refer to the tacit knowledge (\url{https://en.wikipedia.org/wiki/Tacit_knowledge}), which is based on a personal experience.} of their own and life’s nature and exhibit more pro-social behavior as a result, I name ``supersapient'' or ``Homo supersapiens''.

\section*{Homo Supersapiens Hypothesis}

All people want better lives. This idea of ``better'' stems from our beliefs, our conceptual views about the world and the self. In relation to betterment, we also often use the term ``happiness'', which is likewise vague and subjective. These concepts of ``betterment'' and ``happiness'' are just categories created by our minds to estimate how close we are to the point of maximum comfort, which helps us make decisions and act in the conditions of infinite combinations of possible stimuli and responses.

\noindent In the pursuit of happiness we paint a picture of a better life in our imagination. We base it on our past experience and beliefs about what is ``good'', and then we desire to reach it. Of course we want to gauge how effective our efforts are to reach happiness and estimate our progress to know where we are on our imaginary better---worse scale. Especially when we’re regularly asked, ``How are you doing?'' And here’s the difficulty in this: it’s hard to tell if we’re better off now than we were a month or a year ago. We have to compare somehow either our experiences or needs, current and past, but this is not easy for our brains to do. We need a simpler benchmark\thinspace---\thinspace even better if it’s quantifiable because we would only have to add and compare numbers. This might explain why we are so obsessed with money. Money symbolizes the possibility to meet most of our needs, and hence simply by calculating an income we can tell if we do it right and our lives are better then before.\footnote{It is telling that we still measure national and global success with GDP\thinspace---\thinspace an economic indicator, which is expressed as a single number. Even though from the very beginning of its use and to this day many experts point out that it is a poor gauge for social progress.} If the answer is yes, one may feel happier. Here we can add many other easily countable, universal entities, our civilization has created\thinspace---\thinspace these resources are much easier to work with.\footnote{For example, it’s rather hard to measure ideas of respect or authority, but much easier to count and compare ``likes'' or followers on a social network. It’s hard to estimate a person’s food security or level of comfort, but quite easy to count the money, that can provide this. Cars, yachts, real estate, and position on the career ladder all translate into money and can be compared to those of other people.} All these simple symbols of welfare serve us as convenient guides on the path to a better life.

\noindent Let’s now suppose that a person starts to realize the existence of endless life threats (i.e. needs) and comprehends that all feelings (both pleasant and unpleasant) are evoked by them. This leads to the following conclusions and corresponding changes in behavior.\footnote{It’s not about logical reasoning, but rather unconscious decision making such as when we avoid putting a hand in fire. A person with ``fire burns'' knowledge may perceive avoiding fire as a logical choice, yet people who never felt anything hot may find this caution to be nonsense.} So it becomes clear that the route to happiness is not a finite series of steps like ``building a house, planting a tree, fathering a son'', but it is rather a treadmill run. Needs will emerge constantly and satisfying them doesn’t reduce their inflow, so the path to happiness is not that straightforward. Sure, one can try to acquire huge wealth in order to meet every need, but at what cost? Feelings that accompany needs may differ greatly in quality. Often simple pleasures give us more positive emotions than the multiplication of material goods. Often actions that lead to a profit or a satisfied need bring negative emotions, and while we may feel some joy due to a formal success, but the bad aftertaste can negate it completely.

\noindent It turns out that symbols of welfare don’t always go with comfort (i.e., don’t provide welfare). Even the recipe ``more money, less worry'' doesn’t always hold, especially after satisfying basic needs. When a person realizes that conventional values aren’t very effective in bringing about a better life, one starts to search for better values and pay more attention to the quality of their feelings. And feelings become more manageable with this, which helps the search process.

\noindent While a person follows learned cultural patterns (e.g., family, career) for satisfaction of needs in their pursuit of ``happiness'', feelings that come along with it are perceived as externalities that are hard to control. But when the focus is shifted to feelings it becomes clear that prioritizing the satisfaction of needs differently may lead to a greater comfort, and even more\thinspace---\thinspace alternative reactions to threats can be more valuable. Often the usual automatic reactions (like aggression) aren’t the most efficient and thus bring about bad feelings.

\noindent In the search for new, better values, while paying attention to feelings, a person starts to notice how deeply other people influence their emotions, like how many emotions emerge from relationships. This leads to an understanding of how much people influence each other. Someone’s bad mood may turn bad for us, and vice versa. If at the same time a person comprehends human nature, one stops dividing people into good and bad\thinspace---\thinspace we’re all equal and just machines reacting to stimuli, and everyone’s actions are conditioned mostly by current or past circumstances. From these two insights a person begins to reevaluate the importance of others. Taking care of others’ feelings becomes more important than before, and all people start to matter now regardless of their culture. Because we are so conditioned by our surroundings including other people, any positive interaction improves the social environment and increases the likelihood of positive emotions, and vice versa.

\noindent Given these circumstances, a person realizes that working together with others isn’t just more joyful, but it also is more beneficial for everyone involved. Striving for the common good is a better strategy for one’s personal happiness than are hoarding resources for a rainy day or looking for a better life just for oneself. First, working toward the common good makes life better immediately while a ``black-swan'' event may never happen. Even if it does happen, the friends we make in cooperative work are the most versatile resource. Second, improving ones own living conditions while ignoring other people inevitably leads to a stark contrast between one’s cozy small world and the rest of reality, that feels less and less pleasant. It becomes necessary to build borders and barriers, yet they can’t demolish reality, but temporarily conceal it, while the world outside becomes more and more hostile.

\noindent It turns out that cooperation and altruism are beneficial from an egoistic point of view.\footnote{We may consider altruism as the desire to create a positive (in every sense) feedback loop. Improving our shared social conditions leads to greater potential for everyone to experience positive emotions and a further improvement of social conditions.} Cooperative behavior is a part of human nature, but our current living conditions differ significantly from those of our ancestors who developed this behavior. We don’t live in small isolated groups anymore, but in an open, global space relying on money-based economy. We don’t trust strangers, who is the majority of our neighbors these days, but instead entrust our happiness to money. That’s why cooperative strategy often loses to competition.\footnote{We compete with unknown and therefore unimportant people for money and other resources just like we competed with other species and tribes earlier in human history.} Nevertheless, new beliefs lead to greater opportunity for cooperative behavior and decreased conflicts of interest, and thus, diminished aggression (because less conflict means less need for this type of reaction). The fight for survival becomes less like that of the animal world, where needs are satisfied instinctively, and is more sapient and rational, because if we see the big picture, we take into account more important variables.\footnote{The difference between a sapient and a supersapient human is like the difference between a person playing a game for a win and a person playing a game for fun. A win can’t be warranted (memento mori, you will most certainly die) but fun can be. A supersapient human understands that competitive fighting for personal welfare is a rather dubious undertaking, while a cooperative effort for the common good will almost certainly result in a more emotionally joyful life.}

\noindent A society of supersapient individuals will be far less aggressive than today’s society of Homo sapiens. Global issues will be resolving more efficiently because they would be perceived as personal instead of being ignored and deprioritized as they are today.

\section*{Premises of the Hypothesis}

At the beginning of this chapter I mentioned various traditions, which rely on a personal transformative experience. One of these traditions is Buddhism, which probably has accumulated the most practical wisdom on the topic.\footnote{We can assess this by looking at how long the tradition has existed and the amount of texts and practices it’s created.} The hypothesis I outlined above is based partially on my knowledge of the philosophy and practices of Buddhism, and my notion of how it all works in general.

\noindent Among the pieces of experiential knowledge that the Buddhist tradition tries to convey and that are usually referred to as ``insights'', there are three key ones: the truth of impermanence, the truth of suffering and its cause, and the truth of emptiness and no-self. These three insights give a quite robust understanding of some aspects regarding the nature of life and humans. The first is about non-stillness, constant change, and the evolution of all phenomena, which also aligns with our knowledge of the world\thinspace---\thinspace be it stars and atoms, species and habitat, or even our own beliefs. The same principle is applied to needs and feelings: understanding of their ever-changing and inexhaustible nature, along with their causes and effects, constitutes the second insight about suffering. The third truth of emptiness and no-self relates to the understanding that the world and human beings as its part are a system, a machine that works in accordance with some rules. In such a system nothing can exist on its own; everything is conditioned by the mechanics and dynamics of the system. Buddhism emphasizes the idea that all perceived phenomena are just manifestations of the system’s state and don’t exist alone or independently (i.e., they are ``empty''). Particularly, there’s no entity we call self or ``I'' (or a soul, or a free will) that possesses any autonomous good or bad qualities or acts of its own will (i.e., independently of the system). This ``I'' is just a consequence of various causes, and all its actions are just reactions of the machinery.

\noindent Buddhism uses meditation practices as a tool to conceive of these and other ``truths'' with personal experience. By means of meditation practitioners develop the ability to be more mindful about what happens inside oneself and in the outside world. This mindfulness by itself is also a tool for acquiring direct transformative knowledge, not just during meditation, but also in a mundane life.

\noindent During meditation a person gets the chance to see the infinite flux of constantly changing life threats and the feelings they evoke. While mindfulness grows, the constant flux of feelings of everyday life comes into sight as well. At the same time, mindfulness allows one to see personal reactivity and impulsiveness because we often respond automatically even without noticing it. It becomes clear that such automatic reactions, this learned or innate behavior, is often quite inappropriate and leads to more feelings. A person then realizes just how important feelings are because they are both a cause and a result of actions. Looking at feelings and their causes allows one to grasp that all the actions, thoughts, feelings, emotions, and traits of their personality in general stem from their current and past experiences. This realization of causal origination of self from circumstances external to ``I'' leads to the rational conclusion that the implied independent ``I'' or the self with all ascribed good or bad qualities is ``empty'' and just an excess entity in a world view.\footnote{Buddhism uses the concept of karma to represent a buildup of past experience, which determines who a person is and becomes. Using this notion, we can see a considerable difference if we compare two statements and their possible consequences: 1) She did that because she’s a bad person 2) She did that because she’s got bad karma.} And so, with a better understanding of one’s own nature, a person consequently gets a better understanding of other people\thinspace---\thinspace in the same way reactive, conditioned, neither bad nor good. And with that understanding comes the tolerance and compassion so typical of Buddhism.

\noindent The above is a concise outline of my take on how the Buddhist tradition works, molding ordinary people into supersapient beings after years of practice and transformative experiences. Along with this I want to support my thinking with scientific evidence, and below I show some arguments based on experimental data.

\section*{Scientific Evidence}

Scientific studies also provide the foundation for my hypothesis that better comprehension of one’s personal nature results in more pro-social behavior and is the key to positive change in a society. The main premise here is the study of closely related concepts of emotional intelligence, empathy, and perspective-taking.\footnote{Emotional intelligence is the ability to recognize emotions felt by oneself and others and to apply this knowledge. Empathy is the capacity to understand other people’s emotions and their causes. Perspective-taking means to perceive things from someone else’ viewpoint.} Despite confusion in terms and definitions and around their applications (especially for measuring and laboratory research), it is believed that these abilities are tied to one’s capacity to understand the self and to apply this knowledge with regard to other people.\footnote{The basis of this notion is the research into theory of mind (\url{https://en.wikipedia.org/wiki/Theory_of_mind}).} It can be said that people with a better understanding of personal (or human) nature have greater emotional intelligence, empathy, and the ability to consider other viewpoints\thinspace---\thinspace or in other words, be better able to predict other people.

\noindent Research shows that these related abilities correlate with pro-social behavior\footnote{\label{one}Chopik, W. J., O’Brien, E., \& Konrath, S. H. (2017). Differences in Empathic Concern and Perspective Taking Across 63 Countries. \emph{Journal of Cross-Cultural Psychology, 48}(1), 23-38. \url{https://doi.org/10.1177/0022022116673910}} and other benefits. So, for example, they’re associated with stronger social bonds\cref{one}\textsuperscript{,}\footnote{\label{two}Mayer, J. D., Roberts, R. D., \& Barsade, S. G. (2008). Human Abilities: Emotional Intelligence. \emph{Annual Review of Psychology, 59}, 507–536. \url{https://doi.org/10.1146/annurev.psych.59.103006.093646}}, life satisfaction (i.e., happiness)\cref{one}\textsuperscript{,}\cref{two}, and even seem to reduce aggression.\footnote{Day, A., Mohr, P., Howells, K., Gerace, A., \& Lim, L. (2011). The Role of Empathy in Anger Arousal in Violent Offenders and University Students. \emph{International Journal of Offender Therapy and Comparative Criminology, 56}(4), 599–613. \url{https://doi.org/10.1177/0306624x11431061}}\textsuperscript{,}\footnote{Vachon, D. D., \& Lynam, D. R. (2015). Fixing the Problem With Empathy. \emph{Assessment, 23}(2), 135–149. \url{https://doi.org/10.1177/1073191114567941}} With these data taken into account, we may conclude that people whom I refer to as supersapient are happier, better integrated into society, and have a positive impact on their surroundings somehow.

\noindent One recent publication\footnote{Radzvilavicius, A. L., Stewart, A. J., Plotkin J. B. (2019). Evolution of empathetic moral evaluation. \emph{eLife, 8}, e44269. \url{https://doi.org/10.7554/eLife.44269}} argues that greater social inclusion is a result of stronger empathy. People who perceive others more objectively are less prone to be swayed by public opinion and its prejudices, and thus more readily make contact with anyone. And this contact is very important because we are social animals. We have known for long that interaction makes us happier\footnote{There is a vast amount of scientific literature on this, here are some significant papers:
\begin{compactitem}
  \item Kahneman, D., Krueger, A. B., Schkade, D. A., Schwarz, N., \& Stone, A. A. (2004). A Survey Method for Characterizing Daily Life Experience: The Day Reconstruction Method. \emph{Science, 306}(5702), 1776–1780. \url{https://doi.org/10.1126/science.1103572}
  \item Diener, E., \& Seligman, M. E. (2002). Very Happy People. \emph{Psychological Science, 13}(1), 81–84. \url{https://doi.org/10.1111/1467-9280.00415}
  \item Ryan, R. M., \& Deci, E. L. (2001). On Happiness and Human Potentials: A Review of Research on Hedonic and Eudaimonic Well-Being. \emph{Annual Review of Psychology, 52}(1), 141–166. \url{https://doi.org/10.1146/annurev.psych.52.1.141}
\end{compactitem}\vspace{-1.3em}}, we have discovered that happiness spreads through social bonds like a virus.\footnote{Fowler, J. H., \& Christakis, N. A. (2008). Dynamic spread of happiness in a large social network: longitudinal analysis over 20 years in the Framingham Heart Study. \emph{BMJ, 337}, a2338. \url{https://doi.org/10.1136/bmj.a2338}} We see that social inclusion reduces aggression in society and elicits pro-social behavior.\footnote{Abrams, D., Hogg, M. A., \& Marques, J. M. (Eds.). (2005). The social psychology of inclusion and exclusion.}\textsuperscript{,}\footnote{Hamid, N., \& Pretus, C. (2019, June 12). The neuroscience of terrorism: how we convinced a group of radicals to let us scan their brains. \emph{The Conversation}. \url{https://theconversation.com/the-neuroscience-of-terrorism-how-we-convinced-a-group-of-radicals-to-let-us-scan-their-brains-114855}}


\noindent It turns out that a better understanding of oneself and therefore a better understanding of others directly contribute to a better social environment. Supersapient people are not only happier themselves, but facilitate the process of creating happiness of others simply by increasing the average happiness level and the strength of social bonds in a society. And the more we feel happy (i.e., satisfied with life here and now) and the more friends we have, then the less necessary competition and conflicts become. Because satisfaction is the lack of needs and the necessity to fight for resources. And friends are those people, with whom you can cooperatively gather resources while being part of something bigger\thinspace---\thinspace a collective, a community, a planet we can care about together.\footnote{Another study suggests that either feeling oneself a part of something bigger or understanding that everything in the world is interconnected (conditioned and holistic) or maybe both of these strongly correlates with happiness:
\begin{compactitem}
  \item Edinger-Schons, L. M. (2019). Oneness beliefs and their effect on life satisfaction. \emph{Psychology of Religion and Spirituality.} \url{https://doi.org/10.1037/rel0000259}
\end{compactitem}\vspace{-1.3em}}

\section*{The Missing Piece}

As we have observed, it seems that there is a kind of knowledge, universal and accessible to everyone through personal experience\footnote{Which most of people don’t acquire in contrast to the ``fire burns'' knowledge.}, that changes an individual culture for the better. This transformative knowledge can make people less aggressive, more pro-social and happier. As a result, a larger community of such people would diminish the possibility of a global catastrophe because there would be fewer conflicts of interest and more cooperation for the common good. In this section I will illustrate how this utopia can be achieved and what steps must be taken.

\noindent First of all, it’s necessary to note that the process of Homo sapiens transforming into supersapient people has the potential to gain momentum and snowball. It’s because it is more likely for a person to transform from ordinary to supersapient one rather than backwards. It’s like the process of transition to an era of ubiquitous smartphones\thinspace---\thinspace this new way of living is just subjectively better.\footnote{These days it’s better to have a smartphone than have not because it allows one to take advantage of the modern world. Similarly, supersapience gives an advantage because it makes one happier.} Additionally, along with the growth of supersapient population increase cooperation, common welfare, and efforts to positively transform society. At some critical point a positive feedback loop will emerge so that every human becomes supersapient over time.

\noindent However, the process of social transformation requires the ability to create transformative experiences on a faster and greater scale, which we currently lack. If a person had to spend many hours across many years in strange rituals just to acquire a smartphone, we would never live in the smartphone era. Yet it seems that transformative knowledge can be acquired only from such rituals.\footnote{Going back to the fire example, knowledge that fire can burn is easily acquirable, so no one debates about this fact. On the other hand, knowledge about Earth being round is more difficult to build from personal experience. Most of us (not astronauts) just believe this because they rely on expert opinion, though some people even deny this fact.} Not everyone is motivated enough to diligently practice meditation, for example\thinspace---\thinspace it’s hard, confusing, and takes a lot of time (i.e., a precious resource in our fast-paced world).

\noindent If we create an approach for quick and easy acquisition of transformative knowledge, that is simple like buying a phone or putting a hand to a fire, then this will catalyze the process of social change, and the presented utopia of cooperative supersapiens can be realized rather quickly. In order to make this happen we have to understand the mechanics of the transformation, and also to understand what knowledge and experiences are necessary\footnote{Previously I mentioned about empirical knowledge, Buddhist insights and empathy, however, we don’t know scientifically how much all of these concepts have in common just yet. Existing studies don’t provide enough data to confirm how meditation or other practices (e.g., yoga) are related to empathy, or how similar are their effects. Nevertheless, the existence of the relation is almost certain, so we have to map this terra incognita.}, how certain practices lead to those particular experiences, how transformation affects our brain, and how the human brain works in general.

\noindent This field of research is not easy, which is why we still know so little on the subject. Thanks to technological progress, though, we have more and more ways to try to understand the human brain. Although there are still some hurdles (including methodological), research on these topics is gradually gaining momentum. A ``magic pill'' is probably a long way off, but we shouldn’t consider our aim to immediately make everyone super-empathetic or enlightened superheroes like Buddha. Even a small increase of average empathy and mindfulness levels in society\thinspace---\thinspace even by just 10\%\thinspace---\thinspace may bring significant changes and result in a snowball effect having an even greater impact over time. This doesn’t look like a far-off goal because there are already a lot of promising approaches and technologies, and if we put in enough effort we can probably reap positive benefits sooner rather than later.\footnote{I’d like to voice a few ideas that float around so this vision doesn’t look like a pure speculation. There have been interesting studies recently suggesting that it’s possible to improve attention relatively easily: with biofeedback (i), with direct brainwaves adjustment (ii), or even simply by playing a computer game (iii). With attention improved somehow we can try to induce transformative experiences. Perhaps with the help of wearable gadgets this can even be achievable in an everyday routine. Here we could use augmented reality (AR) tech and brain-machine interfaces like Neuralink. Virtual reality (VR) also looks very promising to boost empathy (iv)(v). I hope it’s clear now how helpful high technology can be. So if we can improve effects of meditation with a simple interactive app (vi), just imagine what can be achieved with the use of artificial intelligence (AI), big data, and extensive biofeedback.\\
\indent It’s also worth mentioning the psychedelics here. These substances have being used for centuries to acquire experiences inaccessible in an ordinary state of consciousness. Capable of producing long-term positive psychological effects even with a single intake (vii) these psychoactive substances increasingly draw the attention of scientists. They are also known for their capacity to erase boundaries between one’s self and others, inducing the feeling of connectedness to others and to the natural world, which may likely be a useful tool to boost empathy and altruism. It’s interesting that a recent article (viii) in \textit{The Conversation} expresses a similar idea about how connecting to nature is considered a transformative experience. The author explores the idea that psychedelics, in amplifying this experience, may help in resolving environmental problems of the world.
\begin{compactenum}[i)]
  \item Bagherzadeh, Y., Baldauf, D., Pantazis, D., \& Desimone, R. (2020). Alpha Synchrony and the Neurofeedback Control of Spatial Attention. \emph{Neuron, 105}(3), 577-587. \url{https://doi.org/10.1016/j.neuron.2019.11.001}
  \item Siegle, J. H., Pritchett, D. L., \& Moore, C. I. (2014). Gamma-range synchronization of fast-spiking interneurons can enhance detection of tactile stimuli. \emph{Nature Neuroscience, 17}(10), 1371–1379. \url{https://doi.org/10.1038/nn.3797}
  \item Patsenko, E. G., Adluru, N., Birn, R. M., Stodola, D. E., Kral, T. R., Farajian, R., ... \& Davidson, R. J. (2019) Mindfulness video game improves connectivity of the fronto-parietal attentional network in adolescents: A multi-modal imaging study. \emph{Scientific Reports, 9}(1), 1-8. \url{https://doi.org/10.1038/s41598-019-53393-x}
  \item Slater, M., Neyret, S., Johnston, T., Iruretagoyena, G., de la Campa Crespo, M. Á., Alabèrnia-Segura, M., ... \& Feixas, G. (2019). An experimental study of a virtual reality counselling paradigm using embodied self-dialogue. \emph{Scientific Reports, 9}(1), 1-13. \url{https://doi.org/10.1038/s41598-019-46877-3}
  \item Herrera, F., Bailenson, J., Weisz, E., Ogle, E., \& Zaki, J. (2018). Building long-term empathy: A large-scale comparison of traditional and virtual reality perspective-taking. \emph{PloS one, 13}(10), e0204494. \url{https://doi.org/10.1371/journal.pone.0204494}
  \item Ziegler, D. A., Simon, A. J., Gallen, C. L., Skinner, S., Janowich, J. R., Volponi, J. J., ... \& Gazzaley, A. (2019). Closed-loop digital meditation improves sustained attention in young adults. \emph{Nature Human Behaviour, 3}, 746–757. \url{https://doi.org/10.1038/s41562-019-0611-9}
  \item Nichols, D. E. (2016). Psychedelics. \emph{Pharmacological Reviews, 68}(2), 264–355. \url{https://doi.org/10.1124/pr.115.011478}
  \item Adams, M. (2020, January 28). Could psychedelics help us resolve the climate crisis? \emph{The Conversation}. \url{https://theconversation.com/could-psychedelics-help-us-resolve-the-climate-crisis-129639}
\end{compactenum}\vspace{-1.3em}}

\noindent Of course we may also just simply sit back and wait. But if we want to mitigate the likelihood of catastrophe and increase our own chance of survival as soon as possible (along with creating a better life for ourselves), then it’s in our best interests to be proactive and help science to find the answers we seek. For this we need more neuroscientists, more relevant studies, and more research labs\thinspace---\thinspace all of which requires money. While it may sound trivial, it is also of note that everyone, whether a scientist or not, can contribute to our shared goal. So, private donations can bring additional neuroscience research and training even if major patrons of science (i.e., government and large corporations) don’t support the cause.\footnote{To raise and invest money for the research effectively I propose to create a dedicated fund or institute. Along with resource management, such organization can also develop research guidelines and training programs, do an additional research, interact with businesses and government to attract larger investments and lobby for necessary policy changes.} Everyone concerned with the world’s problems should prioritize donating their resources to develop scientific research and conduct relevant studies.

\noindent Many of the world’s problems share the same cause and the same solution, and when we try to solve them separately it’s like fighting symptoms, rather than the disease. If we want to conquer the ``disease'', however, we have to be more sapient. Be Homo supersapiens.

\end{document}